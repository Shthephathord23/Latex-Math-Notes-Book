% Preamble/Commands/BinaryRelations.tex

\NewDocumentCommand{\Domain}{m}
{
    \mathcal{D}(#1)
}

\NewDocumentCommand{\Range}{m}
{
    \mathcal{R}(#1)
}

\NewDocumentCommand{\Codomain}{m}
{
    \mathcal{C}(#1)
}

\NewDocumentCommand{\Identity}{m}
{
    \mathrm{Id}_{#1}
}

\NewDocumentCommand{\identity}{m}
{
    \mathrm{id}_{#1}
}

\NewDocumentCommand{\UniversalRel}{m o}
{
    \IfValueTF{#2}
    {
        \nabla_{#1, #2}
    }
    {
        \nabla_{#1}
    }
}

\NewDocumentCommand{\ComplementRel}{tp m}
{
    \IfBooleanTF{#1}
    {
        {\left(#2\right)}^\mathrm{c}
    }
    {
        {#2}^\mathrm{c}
    }
}

\NewDocumentCommand{\ConverseRel}{tp m}
{
    \IfBooleanTF{#1}
    {
        {\left(#2\right)}^\mathrm{T}
    }
    {
        {#2}^\mathrm{T}
    }
}




\NewDocumentCommand{\Composition}{m m}
{
    #1 \circ #2
}

\NewDocumentCommand{\Restriction}{m o o}
{
    #1\IfValueT{#2}{\vert_{#2}}\IfValueT{#3}{^{#3}}
}

\NewDocumentCommand{\DirectImage}{m m}
{
    #1\left[#2\right]
}

\NewDocumentCommand{\InverseImage}{m m}
{
    {#1}^{\leftarrow}\left[#2\right]
}
