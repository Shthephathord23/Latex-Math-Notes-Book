\chapter{Binary Relations}

\section{Definition and Basic Properties}

The following are equivalent for any set $R$:

\begin{itemize}
    \item $\Exists v{A}{\Exists vp{B}{R \subseteq A \times B}}$.
    \item $\Forall:{w \in R}{\Exists v{x}{\Exists vp{y}{w = (x, y)}}}$.
\end{itemize}

\begin{proof}
    \textbf{($\Rightarrow$)} Assume $\Exists v{A}{\Exists vp{B}{R \subseteq A \times B}}$.

    Let $w \in R$ be arbitrary. Since $R \subseteq A \times B$, we have $w \in A \times B$. By definition of Cartesian product, $w = (a, b)$ for some $a \in A$ and $b \in B$. 

    Taking $x = a$ and $y = b$, we have shown that $\Forall:{w \in R}{\Exists v{x}{\Exists vp{y}{w = (x, y)}}}$.

    \textbf{($\Leftarrow$)} Assume $\Forall:{w \in R}{\Exists v{x}{\Exists vp{y}{w = (x, y)}}}$.

    We need to find sets $A$ and $B$ such that $R \subseteq A \times B$.

    By the \textbf{Axiom of Replacement}, since for each $w \in R$ there exists a unique first component $x$ such that $w = (x, y)$ for some $y$, we can form the set:
    \[A \defeq \left\{ x \mid \Exists vp{y}{(x, y) \in R} \right\} \]
    % called the \textbf{domain} of $R$.

    Similarly, we can form the set:
    \[B \defeq \left\{ y \mid \Exists vp{x}{(x, y) \in R} \right\} \]
    % called the \textbf{range} of $R$.

    More precisely, by the Axiom of Replacement applied to the function $f(x) = \text{"first component of } x \text{"}$ on the set $R$, we obtain $A$. Similarly for $B$ using the function $g(x) = \text{"second component of } x \text{"}$.

    Now let $w \in R$. By assumption, $\Exists v{x}{\Exists vp{y}{w = (x, y)}}$. By construction of $A$ and $B$ using replacement, we have $x \in A$ and $y \in B$, so $w = (x, y) \in A \times B$.

    Therefore $R \subseteq A \times B$, which proves $\Exists v{A}{\Exists vp{B}{R \subseteq A \times B}}$.
\end{proof}

\begin{remark}
    Notice that the sets $A$ and $B$ are not unique, but they must obey $\Domain{R} \subseteq A$ and $\Range{R} \subseteq B$.
\end{remark}

\begin{definition}[Binary Relations and Graphs of Binary Relations]
    A \textbf{graph of a binary relation} is a set $R$ that satisfies one of the following equivalent conditions above. \\
    A \textbf{binary relation} is a tuple $(A, B, R)$ such that $R \subseteq A \times B$. \\
    We will denote graphs of binary relations with capital letters, such as $R, S$ etc. and binary relations with lowercase letters, such as $r, s$ etc.
\end{definition}

\begin{definition}[Domain and Range for a Graph of a Binary Relation]
    % \hspace{2em}
    Let $R$ be a graph of a binary relation.\\
    The \textbf{domain} of $R$, denoted $\Domain{R}$, is the set:
    \[\Domain{R} \defeq \left\{ x \mid \Exists vp{y}{(x, y) \in R} \right\}\]
    The \textbf{range} of $R$, denoted $\Range{R}$, is the set:
    \[\Range{R} \defeq \left\{ y \mid \Exists vp{x}{(x, y) \in R} \right\}\]
\end{definition}


\subsection{Examples of Binary Relations}

\begin{example}[Identity Relation]
Let $A$ be any set. The \textbf{identity graph} on $A$ is defined as:
\[\Identity{A} \defeq \left\{ (a, a) \mid a \in A \right\}\]
This is the identity relation $\identity{A} \defeq (A, A, \Identity{A})$ where $\Identity{A} \subseteq A \times A$.
\end{example}

\begin{example}[Empty Relation]
Let $A$ and $B$ be any sets. The \textbf{empty relation} from $A$ to $B$ is:
\[\emptyset \subseteq A \times B\]
This gives us the binary relation $(A, B, \emptyset)$.
\end{example}

\begin{example}[Universal Relation]
Let $A$ and $B$ be any sets. The \textbf{universal relation} from $A$ to $B$ is:
\[\UniversalRel{A}[B] \defeq A \times B\]
This gives us the binary relation $(A, B, \UniversalRel{A}[B])$ where every element of $A$ is related to every element of $B$.
\end{example}

\begin{remark}
These three examples represent the extreme cases:
\begin{itemize}
    \item The identity relation relates each element only to itself
    \item The empty relation relates no elements
    \item The universal relation relates every possible pair of elements
\end{itemize}
\end{remark}

\subsection{Operations on Binary Relations}

Since binary relations are defined as tuples $(A, B, R)$ where $R \subseteq A \times B$, we can define set-theoretic operations on them when they share the same domain and codomain.

\begin{definition}[Arbitrary Union of Binary Relations]
Let $\{r_i\}_{i \in I} = \{(A, B, R_i)\}_{i \in I}$ be a family of binary relations with the same domain $A$ and codomain $B$. Their \textbf{union}, denoted $\bigcup_{i \in I} r_i$, is defined as:
\[ \bigcup_{i \in I} r_i \defeq \left( A, B, \bigcup_{i \in I} R_i \right) \]
\end{definition}

\begin{definition}[Arbitrary Intersection of Binary Relations]
Let $\{r_i\}_{i \in I} = \{(A, B, R_i)\}_{i \in I}$ be a family of binary relations with the same domain $A$ and codomain $B$. Their \textbf{intersection}, denoted $\bigcap_{i \in I} r_i$, is defined as:
\[ \bigcap_{i \in I} r_i \defeq \left( A, B, \bigcap_{i \in I} R_i \right) \]
\end{definition}

\begin{definition}[Subset Relation for Binary Relations]
Let $r = (A, B, R)$ and $s = (A, B, S)$ be binary relations with the same domain $A$ and codomain $B$. We say $r \subseteq s$ if and only if $R \subseteq S$.
\end{definition}

\subsection{The Complement of a Binary Relation}
\begin{definition}[Complement of a binary relation]
Let $r = (A, B, R)$ be a binary relation. The \textbf{complement} of $r$, denoted $\ComplementRel{r}$, is the binary relation $(A, B, \ComplementRel{R})$ where:
\[\ComplementRel{R} \defeq \UniversalRel{A}[B] \setminus R = \left\{ (x, y) \in A \times B \mid (x, y) \notin R \right\}\]
\end{definition}

\begin{remark}
Note that the complement operation is only well-defined for binary relations (tuples with specified domain and codomain), not for arbitrary graphs of binary relations, since we need the universal set $A \times B$ to form the complement.
\end{remark}

\begin{proposition}[Properties of Complement for Binary Relations]
Let $\{r_i\}_{i \in I} = \{(A, B, R_i)\}_{i \in I}$ be a family of binary relations with the same domain and codomain. Then:
\begin{enumerate}
    \item \textbf{Involution}: $\ComplementRel p{\ComplementRel{r}} = r$ for any $r = (A, B, R)$
    \item \textbf{De Morgan's Laws}: $\ComplementRel p{\bigcup_{i \in I} r_i} = \bigcap_{i \in I} \ComplementRel{r_i}$ and $\ComplementRel p{\bigcap_{i \in I} r_i} = \bigcup_{i \in I} \ComplementRel{r_i}$
    \item \textbf{Antimonotonicity}: If $r = (A, B, R)$ and $s = (A, B, S)$ with $R \subseteq S$ then $\ComplementRel{s} \subseteq \ComplementRel{r}$
    \item \textbf{Universal and Empty}: The universal relation $(A, B, \UniversalRel{A}[B])$ has complement $(A, B, \emptyset)$, and vice versa
\end{enumerate}
\end{proposition}

\begin{proof}
\textbf{(1) Involution}: 
\begin{align*}
\ComplementRel p{\ComplementRel{r}} &= (A, B, \ComplementRel p{(A \times B) \setminus R}) \\
&= (A, B, (A \times B) \setminus ((A \times B) \setminus R)) \\
&= (A, B, R) = r
\end{align*}

\textbf{(2) De Morgan's Laws}: For the first law:
\begin{align*}
\ComplementRel p{\bigcup_{i \in I} r_i} &= \left(A, B, \ComplementRel p{\bigcup_{i \in I} R_i}\right) \\
&= \left(A, B, (A \times B) \setminus \left(\bigcup_{i \in I} R_i\right)\right) \\
&= \left(A, B, \bigcap_{i \in I} ((A \times B) \setminus R_i)\right) \\
&= \left(A, B, \bigcap_{i \in I} \ComplementRel{R_i}\right) = \bigcap_{i \in I} \ComplementRel{r_i}
\end{align*}
For the second law:
\begin{align*}
\ComplementRel p{\bigcap_{i \in I} r_i} &= \left(A, B, \ComplementRel p{\bigcap_{i \in I} R_i}\right) \\
&= \left(A, B, (A \times B) \setminus \left(\bigcap_{i \in I} R_i\right)\right) \\
&= \left(A, B, \bigcup_{i \in I} ((A \times B) \setminus R_i)\right) \\
&= \left(A, B, \bigcup_{i \in I} \ComplementRel{R_i}\right) = \bigcup_{i \in I} \ComplementRel{r_i}
\end{align*}

\textbf{(3) Antimonotonicity}: Assume $r = (A, B, R)$ and $s = (A, B, S)$ with $R \subseteq S$. Let $(x,y) \in \ComplementRel{S}$. Then $(x,y) \in A \times B$ and $(x,y) \notin S$. Since $R \subseteq S$, if $(x,y)$ were in $R$ then it would be in $S$, which is not the case. Therefore $(x,y) \notin R$, so $(x,y) \in \ComplementRel{R}$.

\textbf{(4) Universal and Empty}: Direct from the definition of complement.
\end{proof}

\subsection{Converse of a Binary Relation}
\begin{definition}[Converse of the graph of a binary relation]
Let $R$ be a graph of a binary relation. The \textbf{converse} (or \textbf{transpose}) of $R$, denoted $\ConverseRel{R}$ or $R^{-1}$, is defined as:
\[\ConverseRel{R} \defeq \left\{ (y, x) \mid (x, y) \in R \right\}\]
\end{definition}

\begin{definition}[Converse of a binary relation]
Let $r = (A, B, R)$ be a binary relation. The \textbf{converse} of $r$, denoted $\ConverseRel{r}$, is the binary relation $(B, A, \ConverseRel{R})$.
\end{definition}

\begin{proposition}[Properties of Converse]
Let $\{R_i\}_{i \in I}$ be a family of graphs of binary relations. Then:
\begin{enumerate}
    \item \textbf{Involution}: $\ConverseRel p{\ConverseRel{R}} = R$ for any graph $R$
    \item \textbf{Domain and Range}: $\Domain{\ConverseRel{R}} = \Range{R}$ and $\Range{\ConverseRel{R}} = \Domain{R}$ for any graph $R$
    \item \textbf{Arbitrary Union}: $\ConverseRel p{\bigcup_{i \in I} R_i} = \bigcup_{i \in I} \ConverseRel{R_i}$
    \item \textbf{Arbitrary Intersection}: $\ConverseRel p{\bigcap_{i \in I} R_i} = \bigcap_{i \in I} \ConverseRel{R_i}$
    \item \textbf{Monotonicity}: If $R \subseteq S$ then $\ConverseRel{R} \subseteq \ConverseRel{S}$
    \item \textbf{Identity}: $\ConverseRel{\Identity{A}} = \Identity{A}$ for any set $A$
    \item \textbf{Empty and Universal}: $\ConverseRel{\emptyset} = \emptyset$ and $\ConverseRel p{\UniversalRel{A}[B]} = \UniversalRel{B}[A]$
\end{enumerate}
\end{proposition}

\begin{proof}
\textbf{(1) Involution}: For any $(x, y)$:
\begin{align*}
(x, y) \in \ConverseRel p{\ConverseRel{R}} &\Leftrightarrow (y, x) \in \ConverseRel{R} \\
&\Leftrightarrow (x, y) \in R
\end{align*}
Therefore $\ConverseRel p{\ConverseRel{R}} = R$.

\textbf{(2) Domain and Range}: 
\begin{align*}
\Domain{\ConverseRel{R}} &= \left\{ x \mid \Exists vp{y}{(x, y) \in \ConverseRel{R}} \right\} \\
&= \left\{ x \mid \Exists vp{y}{(y, x) \in R} \right\} \\
&= \left\{ x \mid x \in \Range{R} \right\} = \Range{R}
\end{align*}
Similarly, $\Range{\ConverseRel{R}} = \Domain{R}$.

\textbf{(3) Arbitrary Union}: For any $(x, y)$:
\begin{align*}
(x, y) \in \ConverseRel p{\bigcup_{i \in I} R_i} &\Leftrightarrow (y, x) \in \bigcup_{i \in I} R_i \\
&\Leftrightarrow \Exists :{i \in I}{(y, x) \in R_i} \\
&\Leftrightarrow \Exists :{i \in I}{(x, y) \in \ConverseRel{R_i}} \\
&\Leftrightarrow (x, y) \in \bigcup_{i \in I} \ConverseRel{R_i}
\end{align*}

\textbf{(4) Arbitrary Intersection}: For any $(x, y)$:
\begin{align*}
(x, y) \in \ConverseRel p{\bigcap_{i \in I} R_i} &\Leftrightarrow (y, x) \in \bigcap_{i \in I} R_i \\
&\Leftrightarrow \Forall :{i \in I}{(y, x) \in R_i} \\
&\Leftrightarrow \Forall :{i \in I}{(x, y) \in \ConverseRel{R_i}} \\
&\Leftrightarrow (x, y) \in \bigcap_{i \in I} \ConverseRel{R_i}
\end{align*}

\textbf{(5) Monotonicity}: Assume $R \subseteq S$. Let $(x, y) \in \ConverseRel{R}$. Then $(y, x) \in R$. Since $R \subseteq S$, we have $(y, x) \in S$, which implies $(x, y) \in \ConverseRel{S}$. Therefore $\ConverseRel{R} \subseteq \ConverseRel{S}$.

\textbf{(6) Identity}: For any $(x, y)$:
\begin{align*}
(x, y) \in \ConverseRel p{\Identity{A}} &\Leftrightarrow (y, x) \in \Identity{A} \\
&\Leftrightarrow y = x \tand y \in A \\
&\Leftrightarrow x = y \tand x \in A \\
&\Leftrightarrow (x, y) \in \Identity{A}
\end{align*}

\textbf{(7) Empty and Universal}: 
For the empty relation: $(x, y) \in \ConverseRel{\emptyset} \Leftrightarrow (y, x) \in \emptyset \Leftrightarrow \text{false} \Leftrightarrow (x, y) \in \emptyset$.

For the universal relation: $(x, y) \in \ConverseRel p{A \times B} \Leftrightarrow (y, x) \in A \times B \Leftrightarrow y \in A \tand x \in B \Leftrightarrow (x, y) \in B \times A$.
\end{proof}

\begin{proposition}[Properties of Converse for Binary Relations]
Let $\{r_i\}_{i \in I} = \{(A, B, R_i)\}_{i \in I}$ be a family of binary relations with the same domain and codomain. Then:
\begin{enumerate}
    \item \textbf{Involution}: $\ConverseRel p{\ConverseRel{r}} = r$ for any relation $r$
    \item \textbf{Domain and Codomain}: $\Domain{\ConverseRel{r}} = \Range{r}$ and $\Codomain{\ConverseRel{r}} = \Domain{r}$ for any relation $r$
    \item \textbf{Arbitrary Union}: $\ConverseRel p{\bigcup_{i \in I} r_i} = \bigcup_{i \in I} \ConverseRel{r_i}$
    \item \textbf{Arbitrary Intersection}: $\ConverseRel p{\bigcap_{i \in I} r_i} = \bigcap_{i \in I} \ConverseRel{r_i}$
    \item \textbf{Monotonicity}: If $r \subseteq s$ then $\ConverseRel{r} \subseteq \ConverseRel{s}$
    \item \textbf{Identity}: $\ConverseRel{\identity{A}} = \identity{A}$ 
    \item \textbf{Empty and Universal}: $\ConverseRel{\emptyset_{A,B}} = \emptyset_{B,A}$ and $\ConverseRel p{\UniversalRel{A}[B]} = \UniversalRel{B}[A]$
\end{enumerate}
\end{proposition}

\begin{proof}
These properties follow directly from the corresponding properties for graphs of binary relations, since the converse operation is defined componentwise on the underlying graph.
\end{proof}

\begin{example}
Let $A = \left\{ 1, 2 \right\}$ and $B = \left\{ a, b \right\}$. Consider the graph of a binary relation:
\[R = \left\{ (1, a), (1, b), (2, a) \right\} \subseteq A \times B\]
Then its converse is:
\[\ConverseRel{R} = \left\{ (a, 1), (b, 1), (a, 2) \right\} \subseteq B \times A\]
Notice that $\Domain{R} = \left\{ 1, 2 \right\} = \Range{\ConverseRel{R}}$ and $\Range{R} = \left\{ a, b \right\} = \Domain{\ConverseRel{R}}$.
\end{example}

\section{Composition of Binary Relations}

\begin{definition}[Composition of graphs of binary relations]
Let $R$ and $S$ be graphs of binary relations. The \textbf{composition} of $R$ and $S$, denoted $\Composition{S}{R}$, is defined as:
\[
\Composition{S}{R} \defeq \left\{ (a, c) \mid \Exists vp{b}{(a, b) \in R \tand (b, c) \in S} \right\}
\]
\end{definition}

\begin{definition}[Composition of binary relations]
Let $r = (A, B, R)$ and $s = (B, C, S)$ be binary relations. The \textbf{composition} of $r$ and $s$, denoted $\Composition{s}{r}$, is the binary relation $(A, C, \Composition{S}{R})$.
\end{definition}

\begin{remark}
Note the order: $\Composition{s}{r}$ means we first apply relation $r$, then relation $s$. This follows the standard function composition convention. \\
The composition of graphs of binary relations is defined for all graphs, but the composition of binary relations is only defined when the codomain of the first relation matches the domain of the second relation.
\end{remark}

\begin{example}
Let $A = \left\{ 1, 2 \right\}$, $B = \left\{ a, b, c \right\}$, and $C = \left\{ x, y \right\}$. Consider:
\begin{align*}
R &= \left\{ (1, a), (1, b), (2, c) \right\} \\
S &= \left\{ (a, x), (b, x), (c, y) \right\}
\end{align*}
Then:
\[\Composition{S}{R} = \left\{ (1, x), (2, y) \right\}\]
\end{example}

\subsection{Basic Properties of Composition}

\begin{proposition}[Basic Properties of Composition for graphs]
Let $R$, $S$, and $T$ be graphs of binary relations. Then:
\begin{enumerate}
    \item $\Domain{\Composition{S}{R}} \subseteq \Domain{R}$
    \item $\Range{\Composition{S}{R}} \subseteq \Range{S}$
    \item $\Composition{S}{R} = \{(a,c) \mid \Exists :  {b \in \Range{R} \cap \Domain{S}}{(a,b) \in R \tand (b,c) \in S}\}$
    \item \textbf{Associativity}: $\Composition{(\Composition{T}{S})}{R} = \Composition{T}{(\Composition{S}{R})}$
    \item \textbf{Identity}: $\Composition{\Identity{B}}{R} = R = \Composition{R}{\Identity{A}} \text{ when } R \subseteq A \times B$
    \item \textbf{Converse of Composition}: $\ConverseRel p{\Composition{S}{R}} = \Composition{\ConverseRel{R}}{\ConverseRel{S}}$
\end{enumerate}
\end{proposition}

\begin{proof}
\textbf{(1)} Let $a \in \Domain{\Composition{S}{R}}$. Then $\Exists{c}{(a,c) \in \Composition{S}{R}}$, so $\Exists{b}{(a,b) \in R \tand (b,c) \in S}$. Thus $a \in \Domain{R}$.

\textbf{(2)} Let $c \in \Range{\Composition{S}{R}}$. Then $\Exists{a}{(a,c) \in \Composition{S}{R}}$, so $\Exists{b}{(a,b) \in R \tand (b,c) \in S}$. Thus $c \in \Range{S}$.

\textbf{(3)} By definition:
\[
\Composition{S}{R} = \left\{ (a,c) \mid \Exists vp{b}{(a,b) \in R \tand (b,c) \in S} \right\}
\]
If $(a,b) \in R$ then $b \in \Range{R}$, and if $(b,c) \in S$ then $b \in \Domain{S}$. Thus:
\[
\Exists vp{b}{(a,b) \in R \tand (b,c) \in S} \Leftrightarrow \Exists :{b \in \Range{R} \cap \Domain{S}}{(a,b) \in R \tand (b,c) \in S}
\]
Therefore the equality holds.

\textbf{(4) Associativity}: 
For any $(a,d)$:
\begin{align*}
(a,d) \in \Composition{(\Composition{T}{S})}{R} 
&\Leftrightarrow \Exists vp{b}{(a,b) \in R \tand (b,d) \in \Composition{T}{S}} \\
&\Leftrightarrow \Exists vp{b}{(a,b) \in R \tand \Exists vp{c}{(b,c) \in S \tand (c,d) \in T}} \\
&\Leftrightarrow \Exists vp{c}{\Exists vp{b}{(a,b) \in R \tand (b,c) \in S} \tand (c,d) \in T} \\
&\Leftrightarrow \Exists vp{c}{(a,c) \in \Composition{S}{R} \tand (c,d) \in T} \\
&\Leftrightarrow (a,d) \in \Composition{T}{(\Composition{S}{R})}
\end{align*}

\textbf{(5) Identity}: 
First part: Assume $R \subseteq A \times B$. For any $(a,b)$:
\begin{align*}
(a,b) \in \Composition{\Identity{A}}{R} 
&\Leftrightarrow \Exists vp{x}{(a,x) \in \Identity{A} \tand (x,b) \in R} \\
&\Leftrightarrow \Exists :p{x \in A}{a = x \tand (x,b) \in R} \\
&\Leftrightarrow (a,b) \in R
\end{align*}
Similarly for $\Composition{R}{\Identity{A}} = R$.

\textbf{(6) Converse of Composition}: 
For any $(a,c)$:
\begin{align*}
(a,c) \in \ConverseRel p{\Composition{S}{R}}
&\Leftrightarrow (c,a) \in \Composition{S}{R} \\
&\Leftrightarrow \Exists vp{b}{(c,b) \in R \tand (b,a) \in S} \\
&\Leftrightarrow \Exists vp{b}{(b,c) \in \ConverseRel{R} \tand (a,b) \in \ConverseRel{S}} \\
&\Leftrightarrow (a,c) \in \Composition{\ConverseRel{R}}{\ConverseRel{S}}
\end{align*}
\end{proof}

\begin{corollary}[Basic Properties of Composition for binary relations]
Let $r$, $s$, and $t$ be binary relations where the compositions are defined. Then:
\begin{enumerate}
    \item \textbf{Associativity}: $\Composition{(\Composition{t}{s})}{r} = \Composition{t}{(\Composition{s}{r})}$
    \item \textbf{Identity}: If $\identity{B} = (B, B, \Identity{B})$ is the identity relation on $B$, then:
    \begin{itemize}
        \item $\Composition{\identity{B}}{r} = r$ for any relation $r = (A, B, R)$
        \item $\Composition{s}{\identity{B}} = s$ for any relation $s = (B, C, S)$
    \end{itemize}
    \item \textbf{Converse of Composition}: $\ConverseRel p{\Composition{s}{r}} = \Composition{\ConverseRel{r}}{\ConverseRel{s}}$
\end{enumerate}
\end{corollary}

\begin{proof}
These properties follow immediately from the corresponding properties for graphs of binary relations, since the composition of binary relations $(A, C, \Composition{S}{R})$ is defined in terms of the composition of their underlying graphs $\Composition{S}{R}$.
\end{proof}

\subsection{Additional Properties of Composition}

\begin{proposition}[Algebraic Properties of Composition for graphs]
Let $R, S, T$ be graphs of binary relations, and let $\{R_i\}_{i \in I}$ and $\{S_i\}_{i \in I}$ be families of graphs of binary relations where the compositions are defined. Then:
\begin{enumerate}
    \item \textbf{Distributivity over Union}:
    \begin{itemize}
        \item Left distributivity: $\Composition{S}{\left( \bigcup_{i \in I} R_i \right)} = \bigcup_{i \in I} \Composition{S}{R_i}$
        \item Right distributivity: $\Composition{\left( \bigcup_{i \in I} S_i \right)}{R} = \bigcup_{i \in I} \Composition{S_i}{R}$
    \end{itemize}
    
    \item \textbf{Subdistributivity over Intersection}:
    \begin{itemize}
        \item Left subdistributivity: $\Composition{S}{\left( \bigcap_{i \in I} R_i \right)} \subseteq \bigcap_{i \in I} \Composition{S}{R_i}$
        \item Right subdistributivity: $\Composition{\left( \bigcap_{i \in I} S_i \right)}{R} \subseteq \bigcap_{i \in I} \Composition{S_i}{R}$
    \end{itemize}
    
    \item \textbf{Monotonicity}:
    \begin{itemize}
        \item If $R_1 \subseteq R_2$ then $\Composition{S}{R_1} \subseteq \Composition{S}{R_2}$ (Left monotonicity)
        \item If $S_1 \subseteq S_2$ then $\Composition{S_1}{R} \subseteq \Composition{S_2}{R}$ (Right monotonicity)
    \end{itemize}
    
    \item \textbf{Antimonotonicity with Complement}:
    \begin{itemize}
        \item $\Composition{\ComplementRel{S}}{R} \subseteq \ComplementRel p{\Composition{S}{R}}$
        \item $\Composition{S}{\ComplementRel{R}} \subseteq \ComplementRel p{\Composition{S}{R}}$
    \end{itemize}
\end{enumerate}
\end{proposition}


\begin{proof}
\textbf{(1) Distributivity over Union}:

For left distributivity:
\begin{align*}
(a,c) \in \Composition{S}{\left( \bigcup_{i \in I} R_i \right)} 
&\Leftrightarrow \Exists vp{b}{(a,b) \in \bigcup_{i \in I} R_i \tand (b,c) \in S} \\
&\Leftrightarrow \Exists vp{b}{\left( \Exists :{i \in I}{(a,b) \in R_i} \right) \tand (b,c) \in S} \\
&\Leftrightarrow \Exists :{i \in I}{\Exists vp{b}{(a,b) \in R_i \tand (b,c) \in S}} \\
&\Leftrightarrow \Exists :{i \in I}{(a,c) \in \Composition{S}{R_i}} \\
&\Leftrightarrow (a,c) \in \bigcup_{i \in I} \Composition{S}{R_i}
\end{align*}

Right distributivity follows similarly.

\textbf{(2) Subdistributivity over Intersection}:

For left subdistributivity:
\begin{align*}
(a,c) \in \Composition{S}{\left( \bigcap_{i \in I} R_i \right)} 
&\Leftrightarrow \Exists vp{b}{(a,b) \in \bigcap_{i \in I} R_i \tand (b,c) \in S} \\
&\Leftrightarrow \Exists vp{b}{\left( \Forall :{i \in I}{(a,b) \in R_i} \right) \tand (b,c) \in S} \\
&\Rightarrow \Forall :{i \in I}{\Exists vp{b}{(a,b) \in R_i \tand (b,c) \in S}} \\
&\Leftrightarrow \Forall :{i \in I}{(a,c) \in \Composition{S}{R_i}} \\
&\Leftrightarrow (a,c) \in \bigcap_{i \in I} \Composition{S}{R_i}
\end{align*}

Right subdistributivity follows similarly. Note that equality doesn't hold in general.

\textbf{(3) Monotonicity}:

If $R_1 \subseteq R_2$:
\begin{align*}
(a,c) \in \Composition{S}{R_1} 
&\Leftrightarrow \Exists vp{b}{(a,b) \in R_1 \tand (b,c) \in S} \\
&\Rightarrow \Exists vp{b}{(a,b) \in R_2 \tand (b,c) \in S} \quad (\text{since } R_1 \subseteq R_2) \\
&\Leftrightarrow (a,c) \in \Composition{S}{R_2}
\end{align*}

If $S_1 \subseteq S_2$:
\begin{align*}
(a,c) \in \Composition{S_1}{R} 
&\Leftrightarrow \Exists vp{b}{(a,b) \in R \tand (b,c) \in S_1} \\
&\Rightarrow \Exists vp{b}{(a,b) \in R \tand (b,c) \in S_2} \quad (\text{since } S_1 \subseteq S_2) \\
&\Leftrightarrow (a,c) \in \Composition{S_2}{R}
\end{align*}

\textbf{(4) Antimonotonicity with Complement}:

For the first part:
\begin{align*}
(a,c) \in \Composition{\ComplementRel {S}}{R} 
&\Leftrightarrow \Exists vp{b}{(a,b) \in R \tand (b,c) \in \ComplementRel {S}} \\
&\Leftrightarrow \Exists vp{b}{(a,b) \in R \tand (b,c) \notin S} \\
&\Rightarrow \neg \Forall vp{b}{(a,b) \in R \Rightarrow (b,c) \in S} \\
&\Leftrightarrow (a,c) \notin \Composition{S}{R} \\
&\Leftrightarrow (a,c) \in \ComplementRel p{\Composition{S}{R}}
\end{align*}

For the second part:
\begin{align*}
(a,c) \in \Composition{S}{\ComplementRel {R}} 
&\Leftrightarrow \Exists vp{b}{(a,b) \in \ComplementRel {R} \tand (b,c) \in S} \\
&\Leftrightarrow \Exists vp{b}{(a,b) \notin R \tand (b,c) \in S} \\
&\Rightarrow \neg \Forall vp{b}{(b,c) \in S \Rightarrow (a,b) \in R} \\
&\Leftrightarrow (a,c) \notin \Composition{S}{R} \\
&\Leftrightarrow (a,c) \in \ComplementRel p{\Composition{S}{R}}
\end{align*}
\end{proof}


\begin{proposition}[Algebraic Properties of Composition for Binary Relations]
Let $r_i = (A, B, R_i)$, $s_i = (B, C, S_i)$, and $t = (C, D, T)$ be binary relations (with matching domains and codomains for composition). Let $\{r_i\}_{i \in I}$ be a family of binary relations from $A$ to $B$, and $\{s_i\}_{i \in I}$ be a family of binary relations from $B$ to $C$. Then:
\begin{enumerate}
    \item \textbf{Distributivity over Union}:
    \begin{itemize}
        \item Left distributivity: $\Composition{s}{\left( \bigcup_{i \in I} r_i \right)} = \bigcup_{i \in I} \Composition{s}{r_i}$
        \item Right distributivity: $\Composition{\left( \bigcup_{i \in I} s_i \right)}{r} = \bigcup_{i \in I} \Composition{s_i}{r}$
    \end{itemize}
    
    \item \textbf{Subdistributivity over Intersection}:
    \begin{itemize}
        \item Left subdistributivity: $\Composition{s}{\left( \bigcap_{i \in I} r_i \right)} \subseteq \bigcap_{i \in I} \Composition{s}{r_i}$
        \item Right subdistributivity: $\Composition{\left( \bigcap_{i \in I} s_i \right)}{r} \subseteq \bigcap_{i \in I} \Composition{s_i}{r}$
    \end{itemize}
    
    \item \textbf{Monotonicity}:
    \begin{itemize}
        \item If $r_1 \subseteq r_2$ then $\Composition{s}{r_1} \subseteq \Composition{s}{r_2}$ (Left monotonicity)
        \item If $s_1 \subseteq s_2$ then $\Composition{s_1}{r} \subseteq \Composition{s_2}{r}$ (Right monotonicity)
    \end{itemize}
    
    \item \textbf{Antimonotonicity with Complement}:
    \begin{itemize}
        \item $\Composition{\ComplementRel{s}}{r} \subseteq \ComplementRel p{\Composition{s}{r}}$
        \item $\Composition{s}{\ComplementRel{r}} \subseteq \ComplementRel p{\Composition{s}{r}}$
    \end{itemize}
    where the complement is taken relative to the same domain and codomain.
\end{enumerate}
\end{proposition}

\begin{proof}
All properties follow from the corresponding properties for the underlying graphs, since composition, union, intersection, and complement of binary relations are defined in terms of their graphs.
\end{proof}

\begin{remark}
The antimonotonicity properties show that composition is "complement-reversing" in each argument. Note that these are inclusions rather than equalities - the reverse inclusions do not hold in general.
\end{remark}

\subsection{Restriction and Corestriction, Direct and Inverse Images}
\begin{definition}[Restriction]
Let $R$ be a graph of a binary relation and $A$ be a set. The \textbf{restriction} of $R$ to $A$, denoted $\Restriction{R}[A]$, is defined as:
\[\Restriction{R}[A] \defeq \left\{ (x, y) \in R \mid x \in A \right\} = R \cap (A \times \Range{R})\]

If $r = (A, B, R)$ is a binary relation, then the restriction of $R$ to $A' \subseteq A$, denoted $\Restriction{r}[A']$, is just $(A', B, \Restriction{R}[A'])$.
\end{definition}

\begin{definition}[Corestriction]
Let $R$ be a graph of a binary relation and $B$ be a set. The \textbf{corestriction} of $R$ to $B$, denoted $\Restriction{R}[][B]$, is defined as:
\[\Restriction{R}[][B] \defeq \left\{ (x, y) \in R \mid y \in B \right\} = R \cap (\Domain{R} \times B)\]

If $r = (A, B, R)$ is a binary relation, then the corestriction of $R$ to $B' \subseteq B$, denoted $\Restriction{r}[][B']$, is just $(A, B', \Restriction{R}[][B'])$.
\end{definition}

\begin{remark}
    The notation works to both restrict and corestrict at the same time.
    \[ \Restriction{R}[A][B] \defeq \left\{ (x, y) \in R \mid x \in A \tand y \in B \right\} = R \cap (A \times B) \]

    If $r = (A, B, R)$ is a binary relation and $A' \subseteq A$ and $B' \subseteq B$, denoted $\Restriction{r}[A'][B']$, is just $(A', B', \Restriction{R}[A'][B'])$.
\end{remark}

\begin{definition}[Direct Image]
Let $R$ be a graph of a binary relation and $A$ be a set. The \textbf{direct image} of $A$ under $R$, denoted $\DirectImage{R}{A}$, is defined as:
\[ \DirectImage{R}{A} \defeq \left\{ y \in \Range{R} \mid \Exists :{x \in A}{(x, y) \in R} \right\} = \Range{\Restriction{R}[A][]} \]

If $r= (A, B, R)$ is a binary relation, then the direct image of $A$ under $r$, denoted $\DirectImage{r}{A}$, is just $\DirectImage{r}{A} \defeq \DirectImage{R}{A}$.
\end{definition}

\begin{definition}[Inverse Image]
Let $R$ be a graph of a binary relation and $B$ be a set. The \textbf{inverse image} of $B$ under $R$, denoted $\InverseImage{R}{B}$, is defined as:
\[ \InverseImage{R}{B} \defeq \left\{ x \in \Domain{R} \mid \Exists :{y \in B}{(x, y) \in R} \right\} = \Domain{\Restriction{R}[][B]} \]

If $r = (A, B, R)$ is a binary relation, then the inverse image of $B$ under $r$, denoted $\InverseImage{r}{B}$, is just $\InverseImage{r}{B} \defeq \InverseImage{R}{B}$.
\end{definition}

\begin{remark}[Composition via Restriction]
For $r = (A,B,R)$, $s = (C,D,S)$ with $B \subseteq C$, we can define:
\[ \Composition{s}{r} \defeq (A,D,\Composition{(\Restriction{S}[\Range{R}][D])}{(\Restriction{R}[A][\Domain{S}])}) \]
This equals $(A,D,\Composition{S}{R})$ since:
\[ \Composition{S}{R} = \Composition{(\Restriction{S}[\Range{R}][D])}{(\Restriction{R}[A][\Domain{S}])} \]
\end{remark}


\begin{proposition}[Unification via Composition]
Let $R$ be a graph of a binary relation and let $A, B$ be sets. Then:
\begin{enumerate}
    \item $\Restriction{R}[A] = \Composition{R}{\Identity{A}}$
    \item $\Restriction{R}[][B] = \Composition{\Identity{B}}{R}$
    \item $\DirectImage{R}{A} = \Range{\Composition{R}{\Identity{A}}}$
    \item $\InverseImage{R}{B} = \Domain{\Composition{\Identity{B}}{R}}$
\end{enumerate}
\end{proposition}

\begin{proof}
\textbf{(1) Restriction}: 
\begin{align*}
\Composition{R}{\Identity{A}} &= \left\{ (x, z) \mid \Exists vp{y}{(x, y) \in \Identity{A} \tand (y, z) \in R} \right\} \\
&= \left\{ (x, z) \mid \Exists :{y \in A}{x = y \tand (y, z) \in R} \right\} \\
&= \left\{ (x, z) \mid x \in A \tand (x, z) \in R \right\} \\
&= \Restriction{R}[A]
\end{align*}

\textbf{(2) Corestriction}: 
\begin{align*}
\Composition{\Identity{B}}{R} &= \left\{ (x, z) \mid \Exists vp{y}{(x, y) \in R \tand (y, z) \in \Identity{B}} \right\} \\
&= \left\{ (x, z) \mid \Exists :{y \in B}{(x, y) \in R \tand y = z} \right\} \\
&= \left\{ (x, y) \mid (x, y) \in R \tand y \in B \right\} \\
&= \Restriction{R}[][B]
\end{align*}

\textbf{(3) Direct Image}: From property (1):
\begin{align*}
\DirectImage{R}{A} &= \left\{ y \mid \Exists :{x \in A}{(x, y) \in R} \right\} \\
&= \Range{\Restriction{R}[A]} \\
&= \Range{\Composition{R}{\Identity{A}}}
\end{align*}

\textbf{(4) Inverse Image}: From property (2):
\begin{align*}
\InverseImage{R}{B} &= \left\{ x \mid \Exists :{y \in B}{(x, y) \in R} \right\} \\
&= \Domain{\Restriction{R}[][B]} \\
&= \Domain{\Composition{\Identity{B}}{R}}
\end{align*}
\end{proof}

\begin{corollary}[Duality via Converse]
The operations of restriction and corestriction are dual via converse, as are direct and inverse images:
\begin{enumerate}
    \item $\Restriction{\ConverseRel{R}}[B] = \ConverseRel p{\Restriction{R}[][B]}$
    \item $\Restriction{\ConverseRel{R}}[][A] = \ConverseRel p{\Restriction{R}[A]}$
    \item $\Restriction{R}[][B] = \ConverseRel p{\Restriction{\ConverseRel{R}}[B]}$
    \item $\Restriction{R}[A] = \ConverseRel p{\Restriction{\ConverseRel{R}}[][A]}$
    \item $\DirectImage{\ConverseRel{R}}{A} = \InverseImage{R}{A}$
    \item $\InverseImage{\ConverseRel{R}}{B} = \DirectImage{R}{B}$
    \item $\DirectImage{R}{A} = \InverseImage{\ConverseRel{R}}{A}$
    \item $\InverseImage{R}{B} = \DirectImage{\ConverseRel{R}}{B}$
\end{enumerate}
\end{corollary}

\begin{proof}
These follow directly from the definitions and properties of converse operations:

\textbf{(1)}: By definition of restriction and converse:
\begin{align*}
\Restriction{\ConverseRel{R}}[B] &= \left\{ (x, y) \in \ConverseRel{R} \mid x \in B \right\} \\
&= \left\{ (x, y) \mid (y, x) \in R \tand x \in B \right\} \\
&= \ConverseRel{\left\{ (y, x) \mid (y, x) \in R \tand x \in B \right\}} \\
&= \ConverseRel p{\Restriction{R}[][B]}
\end{align*}

\textbf{(2)}: By definition of corestriction and converse:
\begin{align*}
\Restriction{\ConverseRel{R}}[][A] &= \left\{ (x, y) \in \ConverseRel{R} \mid y \in A \right\} \\
&= \left\{ (x, y) \mid (y, x) \in R \tand y \in A \right\} \\
&= \ConverseRel{\left\{ (y, x) \mid (y, x) \in R \tand y \in A \right\}} \\
&= \ConverseRel p{\Restriction{R}[A]}
\end{align*}

\textbf{(3)} and \textbf{(4)}: These follow from (1) and (2) by applying converse to both sides and using the involution property $\ConverseRel p{\ConverseRel{R}} = R$.

\textbf{(5)}: By definition:
\begin{align*}
\DirectImage{\ConverseRel{R}}{A} &= \left\{ y \mid \Exists :{x \in A}{(x, y) \in \ConverseRel{R}} \right\} \\
&= \left\{ y \mid \Exists :{x \in A}{(y, x) \in R} \right\} \\
&= \InverseImage{R}{A}
\end{align*}

\textbf{(6)}: By definition:
\begin{align*}
\InverseImage{\ConverseRel{R}}{B} &= \left\{ x \mid \Exists :{y \in B}{(x, y) \in \ConverseRel{R}} \right\} \\
&= \left\{ x \mid \Exists :{y \in B}{(y, x) \in R} \right\} \\
&= \DirectImage{R}{B}
\end{align*}

\textbf{(7)} and \textbf{(8)}: These follow from (5) and (6) by applying converse to both sides and using the involution property $\ConverseRel p{\ConverseRel{R}} = R$.
\end{proof}

\begin{remark}
This proposition shows that restriction, corestriction, direct images, and inverse images are all special cases of composition with identity relations. This unifies these seemingly different operations under the single framework of relation composition, demonstrating the fundamental role of composition in binary relation theory. 

The corollary further shows the elegant duality between these operations via the converse operation, illustrating the deep symmetries inherent in the theory of binary relations. Specifically, corestriction is the restriction of the converse and dually, while direct image is the inverse image of the converse and dually.
\end{remark}

\begin{corollary}[Properties of Domain and Range]
Let $\{R_i\}_{i \in I}$ be a family of graphs of binary relations. Then:
\begin{enumerate}
    \item \textbf{Distributivity over union}:
        \begin{itemize}
            \item $\Domain{\bigcup_{i \in I} R_i} = \bigcup_{i \in I} \Domain{R_i}$
            \item $\Range{\bigcup_{i \in I} R_i} = \bigcup_{i \in I} \Range{R_i}$
        \end{itemize}
    \item \textbf{Subdistributivity over intersection}:
        \begin{itemize}
            \item $\Domain{\bigcap_{i \in I} R_i} \subseteq \bigcap_{i \in I} \Domain{R_i}$
            \item $\Range{\bigcap_{i \in I} R_i} \subseteq \bigcap_{i \in I} \Range{R_i}$
        \end{itemize}
\end{enumerate}
\end{corollary}

\begin{proof}
\textbf{(1) Distributivity over union}:
\begin{align*}
x \in \Domain{\bigcup_{i \in I} R_i} 
&\iff \Exists :p{y}{(x, y) \in \bigcup_{i \in I} R_i} \\
&\iff \Exists :p{y}{\Exists :{i \in I}{(x, y) \in R_i}} \\
&\iff \Exists :{i \in I}{\Exists :p{y}{(x, y) \in R_i}} \\
&\iff \Exists :{i \in I}{x \in \Domain{R_i}} \\
&\iff x \in \bigcup_{i \in I} \Domain{R_i}
\end{align*}
The proof for range follows from $\Range{\bigcup_{i \in I} R_i} = \Domain{\ConverseRel{\bigcup_{i \in I} R_i}} = \Domain{\bigcup_{i \in I} \ConverseRel{R_i}} = \bigcup_{i \in I} \Domain{\ConverseRel{R_i}} = \bigcup_{i \in I} \Range{R_i}$.

\textbf{(2) Subdistributivity over intersection}:
\begin{align*}
x \in \Domain{\bigcap_{i \in I} R_i} 
&\iff \Exists :p{y}{(x, y) \in \bigcap_{i \in I} R_i} \\
&\iff \Exists :p{y}{\Forall :{i \in I}{(x, y) \in R_i}} \\
&\implies \Forall :{i \in I}{\Exists :p{y}{(x, y) \in R_i}} \\
&\iff \Forall :{i \in I}{x \in \Domain{R_i}} \\
&\iff x \in \bigcap_{i \in I} \Domain{R_i}
\end{align*}
The proof for range follows from $\Range{\bigcap_{i \in I} R_i} = \Domain{\ConverseRel{\bigcap_{i \in I} R_i}} = \Domain{\bigcap_{i \in I} \ConverseRel{R_i}} \subseteq \bigcap_{i \in I} \Domain{\ConverseRel{R_i}} = \bigcap_{i \in I} \Range{R_i}$.
\end{proof}

\begin{corollary}[Properties of Direct and Inverse Images]
Let $R$ and $S$ be graphs of binary relations, and let $A, B$ be sets. Then:
\begin{enumerate}
    \item \textbf{Monotonicity in the relation}:
        \begin{itemize}
            \item If $R \subseteq S$ then $\DirectImage{R}{A} \subseteq \DirectImage{S}{A}$
            \item If $R \subseteq S$ then $\InverseImage{R}{B} \subseteq \InverseImage{S}{B}$
        \end{itemize}
    \item \textbf{Monotonicity in the set}:
        \begin{itemize}
            \item If $A_1 \subseteq A_2$ then $\DirectImage{R}{A_1} \subseteq \DirectImage{R}{A_2}$
            \item If $B_1 \subseteq B_2$ then $\InverseImage{R}{B_1} \subseteq \InverseImage{R}{B_2}$
        \end{itemize}
    \item \textbf{Distributivity over union of sets}:
        \begin{itemize}
            \item $\DirectImage{R}{\left( \bigcup_{i \in I} A_i \right)} = \bigcup_{i \in I} \DirectImage{R}{A_i}$
            \item $\InverseImage{R}{\left( \bigcup_{i \in I} B_i \right)} = \bigcup_{i \in I} \InverseImage{R}{B_i}$
        \end{itemize}
    \item \textbf{Subdistributivity over intersection of sets}:
        \begin{itemize}
            \item $\DirectImage{R}{\left( \bigcap_{i \in I} A_i \right)} \subseteq \bigcap_{i \in I} \DirectImage{R}{A_i}$
            \item $\InverseImage{R}{\left( \bigcap_{i \in I} B_i \right)} \subseteq \bigcap_{i \in I} \InverseImage{R}{B_i}$
        \end{itemize}
    \item \textbf{Distributivity over union of relations}:
        \begin{itemize}
            \item $\DirectImage{\left( \bigcup_{i \in I} R_i \right)}{A} = \bigcup_{i \in I} \DirectImage{R_i}{A}$
            \item $\InverseImage{\left( \bigcup_{i \in I} R_i \right)}{B} = \bigcup_{i \in I} \InverseImage{R_i}{B}$
        \end{itemize}
    \item \textbf{Subdistributivity over intersection of relations}:
        \begin{itemize}
            \item $\DirectImage{\left( \bigcap_{i \in I} R_i \right)}{A} \subseteq \bigcap_{i \in I} \DirectImage{R_i}{A}$
            \item $\InverseImage{\left( \bigcap_{i \in I} R_i \right)}{B} \subseteq \bigcap_{i \in I} \InverseImage{R_i}{B}$
        \end{itemize}
    \item \textbf{Antimonotonicity with complement}:
        \begin{itemize}
            \item $\DirectImage{\ComplementRel{R}}{A} \subseteq \ComplementRel{\DirectImage{R}{A}}$
            \item $\InverseImage{\ComplementRel{R}}{B} \subseteq \ComplementRel{\InverseImage{R}{B}}$
        \end{itemize}
\end{enumerate}
\end{corollary}

\begin{proof}
All properties follow from the representation of images as compositions with identity relations and the algebraic properties of composition.

\textbf{(1) Monotonicity in the relation}:
From the representation $\DirectImage{R}{A} = \Range{\Composition{R}{\Identity{A}}}$ and the right monotonicity of composition, if $R \subseteq S$ then $\Composition{R}{\Identity{A}} \subseteq \Composition{S}{\Identity{A}}$, so their ranges satisfy $\DirectImage{R}{A} \subseteq \DirectImage{S}{A}$. Similarly for inverse images using $\InverseImage{R}{B} = \Domain{\Composition{\Identity{B}}{R}}$ and left monotonicity.

\textbf{(2) Monotonicity in the set}:
If $A_1 \subseteq A_2$, then $\Identity{A_1} \subseteq \Identity{A_2}$. By left monotonicity of composition, $\Composition{R}{\Identity{A_1}} \subseteq \Composition{R}{\Identity{A_2}}$, so their ranges satisfy $\DirectImage{R}{A_1} \subseteq \DirectImage{R}{A_2}$. Similarly for inverse images using right monotonicity.

\textbf{(3) Distributivity over union of sets}:
\begin{align*}
\DirectImage{R}{\left( \bigcup_{i \in I} A_i \right)} 
&= \Range{\Composition{R}{\Identity{\bigcup_{i} A_i}}} \\
&= \Range{\Composition{R}{\left( \bigcup_{i} \Identity{A_i} \right)}} \\
&= \Range{\bigcup_{i} \Composition{R}{\Identity{A_i}}} \\
&= \bigcup_{i} \Range{\Composition{R}{\Identity{A_i}}} \\
&= \bigcup_{i} \DirectImage{R}{A_i}
\end{align*}
Similarly for inverse images using left distributivity of composition.

\textbf{(4) Subdistributivity over intersection of sets}:
\begin{align*}
\DirectImage{R}{\left( \bigcap_{i \in I} A_i \right)} 
&= \Range{\Composition{R}{\Identity{\bigcap_{i} A_i}}} \\
&= \Range{\Composition{R}{\left( \bigcap_{i} \Identity{A_i} \right)}} \\
&\subseteq \Range{\bigcap_{i} \Composition{R}{\Identity{A_i}}} \\
&\subseteq \bigcap_{i} \Range{\Composition{R}{\Identity{A_i}}} \\
&= \bigcap_{i} \DirectImage{R}{A_i}
\end{align*}
Similarly for inverse images using left subdistributivity of composition.

\textbf{(5) Distributivity over union of relations}:
\begin{align*}
\DirectImage{\left( \bigcup_{i \in I} R_i \right)}{A} 
&= \Range{\Composition{\left( \bigcup_{i} R_i \right)}{\Identity{A}}} \\
&= \Range{\bigcup_{i} \Composition{R_i}{\Identity{A}}} \\
&= \bigcup_{i} \Range{\Composition{R_i}{\Identity{A}}} \\
&= \bigcup_{i} \DirectImage{R_i}{A}
\end{align*}
Similarly for inverse images using right distributivity.

\textbf{(6) Subdistributivity over intersection of relations}:
\begin{align*}
\DirectImage{\left( \bigcap_{i \in I} R_i \right)}{A} 
&= \Range{\Composition{\left( \bigcap_{i} R_i \right)}{\Identity{A}}} \\
&\subseteq \Range{\bigcap_{i} \Composition{R_i}{\Identity{A}}} \\
&\subseteq \bigcap_{i} \Range{\Composition{R_i}{\Identity{A}}} \\
&= \bigcap_{i} \DirectImage{R_i}{A}
\end{align*}
Similarly for inverse images using right subdistributivity.

\textbf{(7) Antimonotonicity with complement}:
From the antimonotonicity of composition:
\begin{align*}
\DirectImage{\ComplementRel{R}}{A} 
&= \Range{\Composition{\ComplementRel{R}}{\Identity{A}}} \\
&\subseteq \Range{\ComplementRel p{\Composition{R}{\Identity{A}}}} \\
&\subseteq \ComplementRel{\Range{\Composition{R}{\Identity{A}}}} \\
&= \ComplementRel{\DirectImage{R}{A}}
\end{align*}
Similarly for inverse images using the other antimonotonicity property of composition. Note that the last inclusion holds because for any set $X$, $\Range{\ComplementRel{X}} \subseteq \ComplementRel{\Range{X}}$ when the complement is taken relative to a universal set.
\end{proof}



\begin{proposition}[Properties of Images for Binary Relations]
Let $r = (A, B, R)$ and $s = (A, B, S)$ be binary relations with the same domain and codomain. Then:
\begin{enumerate}
    \item \textbf{Monotonicity in the relation}:
        \begin{itemize}
            \item If $R \subseteq S$ then $\DirectImage{r}{A} \subseteq \DirectImage{s}{A}$
            \item If $R \subseteq S$ then $\InverseImage{r}{B} \subseteq \InverseImage{s}{B}$
        \end{itemize}
    \item \textbf{Monotonicity in the set}:
        \begin{itemize}
            \item If $A_1 \subseteq A_2$ then $\DirectImage{r}{A_1} \subseteq \DirectImage{r}{A_2}$
            \item If $B_1 \subseteq B_2$ then $\InverseImage{r}{B_1} \subseteq \InverseImage{r}{B_2}$
        \end{itemize}
    \item \textbf{Distributivity over union of sets}:
        \begin{itemize}
            \item $\DirectImage{r}{\left( \bigcup_{i \in I} A_i \right)} = \bigcup_{i \in I} \DirectImage{r}{A_i}$
            \item $\InverseImage{r}{\left( \bigcup_{i \in I} B_i \right)} = \bigcup_{i \in I} \InverseImage{r}{B_i}$
        \end{itemize}
    \item \textbf{Subdistributivity over intersection of sets}:
        \begin{itemize}
            \item $\DirectImage{r}{\left( \bigcap_{i \in I} A_i \right)} \subseteq \bigcap_{i \in I} \DirectImage{r}{A_i}$
            \item $\InverseImage{r}{\left( \bigcap_{i \in I} B_i \right)} \subseteq \bigcap_{i \in I} \InverseImage{r}{B_i}$
        \end{itemize}
    \item \textbf{Distributivity over union of relations}:
        \begin{itemize}
            \item $\DirectImage{\left( \bigcup_{i \in I} r_i \right)}{A} = \bigcup_{i \in I} \DirectImage{r_i}{A}$
            \item $\InverseImage{\left( \bigcup_{i \in I} r_i \right)}{B} = \bigcup_{i \in I} \InverseImage{r_i}{B}$
        \end{itemize}
    \item \textbf{Subdistributivity over intersection of relations}:
        \begin{itemize}
            \item $\DirectImage{\left( \bigcap_{i \in I} r_i \right)}{A} \subseteq \bigcap_{i \in I} \DirectImage{r_i}{A}$
            \item $\InverseImage{\left( \bigcap_{i \in I} r_i \right)}{B} \subseteq \bigcap_{i \in I} \InverseImage{r_i}{B}$
        \end{itemize}
    \item \textbf{Antimonotonicity with complement}:
        \begin{itemize}
            \item $\DirectImage{\ComplementRel{r}}{A} \subseteq B \setminus \DirectImage{r}{A}$
            \item $\InverseImage{\ComplementRel{r}}{B} \subseteq A \setminus \InverseImage{r}{B}$
        \end{itemize}
\end{enumerate}
\end{proposition}

\begin{proof}
The first six properties follow directly from the corresponding properties for graphs of binary relations. For the antimonotonicity properties:

Let $r = (A, B, R)$. Then $\ComplementRel{r} = (A, B, (A \times B) \setminus R)$.

For direct image:
\begin{align*}
\DirectImage{\ComplementRel{r}}{A} 
&= \left\{ y \in B \mid \Exists :{x \in A}{(x, y) \in (A \times B) \setminus R} \right\} \\
&= \left\{ y \in B \mid \Exists :{x \in A}{(x, y) \notin R} \right\} \\
&\subseteq \left\{ y \in B \mid y \notin \DirectImage{r}{A} \right\} \\
&= B \setminus \DirectImage{r}{A}
\end{align*}

Similarly for inverse image:
\begin{align*}
\InverseImage{\ComplementRel{r}}{B} 
&= \left\{ x \in A \mid \Exists :{y \in B}{(x, y) \in (A \times B) \setminus R} \right\} \\
&= \left\{ x \in A \mid \Exists :{y \in B}{(x, y) \notin R} \right\} \\
&\subseteq \left\{ x \in A \mid x \notin \InverseImage{r}{B} \right\} \\
&= A \setminus \InverseImage{r}{B}
\end{align*}
\end{proof}



















\chapter{Heterogenous Binary Relations}

\section{Definition}

Let $r = (A, B, R)$ be a binary relation. If $A = B$, we call $r$ a homogenous relation. If we do not care whether $A = B$ we say $r$ is heterogenous.

\section{Classification}

We classify binary relations based on how they relate elements from their domain to their codomain. The following four fundamental properties capture different aspects of this relationship.

\begin{definition}[Univalent Relation]
Let $r = (A, B, R)$ be a binary relation. We say $r$ is \textbf{univalent} (or \textbf{right-unique}) if every element in the domain is related to at most one element in the codomain:
\[ \Forall :{a \in A}{\Forall :{b_1, b_2 \in B}{(a, b_1) \in R \tand (a, b_2) \in R \implies b_1 = b_2}} \]
Equivalently, if $(a, b_1) \in R$ and $(a, b_2) \in R$, then $b_1 = b_2$. \\
If $r = (A, B, R)$ we also say the graph of binary relations $R$ is univalent.
\end{definition}

\begin{definition}[Total Relation]
Let $r = (A, B, R)$ be a binary relation. We say $r$ is \textbf{total} (or sometimes \textbf{left-total}) if every element in the domain $A$ is related to at least one element in the codomain $B$:
\[ \Forall :{a \in A}{\Exists :{b \in B}{(a, b) \in R}} \]
Equivalently, $\Domain{R} = A$.
\end{definition}

\begin{definition}[Injective Relation]
Let $r = (A, B, R)$ be a binary relation. We say $r$ is \textbf{injective} (or \textbf{left-unique}) if every element in the codomain is related to at most one element in the domain:
\[ \Forall :{b \in B}{\Forall :{a_1, a_2 \in A}{(a_1, b) \in R \tand (a_2, b) \in R \implies a_1 = a_2}} \]
Equivalently, if $(a_1, b) \in R$ and $(a_2, b) \in R$, then $a_1 = a_2$.
\end{definition}

\begin{definition}[Surjective Relation]
Let $r = (A, B, R)$ be a binary relation. We say $r$ is \textbf{surjective} (or \textbf{right-total}) if every element in the codomain $B$ is related to at least one element in the domain $A$:
\[ \Forall :{b \in B}{\Exists p{a \in A}{(a, b) \in R}} \]
Equivalently, $\Range{R} = B$.
\end{definition}

\begin{definition}[Functions as Relation]
Let $r = (A, B, R)$ be a binary relation. We say $r$ is a \textbf{function} if it is both total and univalent. That is:
\begin{enumerate}
    \item Every element in $A$ is related to at least one element in $B$ (total)
    \item Every element in $A$ is related to at most one element in $B$ (univalent)
\end{enumerate}
Equivalently, $r$ is a function $\iff \Forall :{a \in A}{\ExistsUnique {b \in B}{(a, b) \in R}}$.
\end{definition}

\begin{remark}
    If $r = (A, B, R)$ we also say the graph of binary relations $R$ is univalent.
    If $r = (A, B, R)$ we also say the graph of binary relations $R$ is injective.
    We do not say a graph of a binary relation is total or surjective since more information on the domain/codomain is needed.
\end{remark}

\begin{remark}
Terminology:\\
These properties can be combined in various ways:
\begin{itemize}
    \item A relation that is total but not univalent relates each element of $A$ to one or more elements of $B$
    \item A relation that is univalent but not total relates some (but not necessarily all) elements of $A$ to exactly one element of $B$ (this is sometimes called a \textbf{partial function})
    \item A relation that is both a function and injective is called an \textbf{injective function}
    \item A relation that is both a function and surjective is called a \textbf{surjective function}
    \item A relation that is a function, injective, and surjective is called a \textbf{bijective function} or \textbf{bijection}
    \item A relation that is surjective but not a function can relate elements of $A$ to multiple elements of $B$, as long as every element of $B$ is covered
\end{itemize}
\end{remark}

\begin{example}[Basic Examples]
Let $A$ be any nonempty set.
\begin{enumerate}
    \item The \textbf{identity relation} $\identity{A} = (A, A, \Identity{A})$ is total, univalent, injective, and surjective (hence a bijection).
    \item The \textbf{empty relation} $(A, B, \emptyset)$ is univalent and injective, but not total (unless $A = \emptyset$) and not surjective (unless $B = \emptyset$).
    \item The \textbf{universal relation} $(A, B, A \times B)$ is total and surjective, but not univalent (unless $|A| \leq 1$) and not injective (unless $|B| \leq 1$).
\end{enumerate}
\end{example}

\begin{example}
Let $A = \{1, 2, 3\}$ and $B = \{a, b, c\}$.

\begin{enumerate}
    \item $R_1 = \{(1, a), (2, b), (3, c)\}$ is total, univalent, injective, surjective, hence a bijective function.
    \item $R_2 = \{(1, a), (1, b), (2, a), (3, c)\}$ is total and surjective, but not univalent (hence not a function) and not injective.
    \item $R_3 = \{(1, a), (2, b)\}$ is univalent and injective but not total and not surjective (hence not a function).
    \item $R_4 = \{(1, a), (2, a), (3, b)\}$ is total and univalent (hence a function) but not injective and not surjective.
    \item $R_5 = \{(1, a), (2, a), (2, b)\}$ is neither total nor univalent nor injective nor surjective nor a function.
    \item $R_6 = \{(1, a), (2, b), (3, c), (3, a)\}$ is total and surjective but not univalent (hence not a function) and not injective.
    \item $R_7 = \{(1, a), (1, b), (2, b), (3, c)\}$ is total, univalent, and surjective (hence a surjective function) but not injective.
\end{enumerate}
\end{example}

\begin{proposition}[Characterization via Converse]
Let $r = (A, B, R)$ be a binary relation. Then:
\begin{enumerate}
    \item $r$ is injective if and only if $\ConverseRel{r}$ is univalent
    \item $r$ is univalent if and only if $\ConverseRel{r}$ is injective
    \item $r$ is surjective if and only if $\ConverseRel{r}$ is total
    \item $r$ is total if and only if $\ConverseRel{r}$ is surjective
\end{enumerate}
\end{proposition}

\begin{proof}
\textbf{(1)} $r$ is injective means:
\[ \Forall :{b \in B}{\Forall :{a_1, a_2 \in A}{(a_1, b) \in R \tand (a_2, b) \in R \implies a_1 = a_2}} \]

This is equivalent to:
\[ \Forall :{b \in B}{\Forall :{a_1, a_2 \in A}{(b, a_1) \in \ConverseRel{R} \tand (b, a_2) \in \ConverseRel{R} \implies a_1 = a_2}} \]

which means $\ConverseRel{r}$ is univalent.

\textbf{(2)} This follows by applying (1) to $\ConverseRel{r}$ and using the involution property $\ConverseRel p{\ConverseRel{r}} = r$.

\textbf{(3)} $r$ is surjective means:
\[ \Forall :{b \in B}{\Exists p{a \in A}{(a, b) \in R}} \]

This is equivalent to:
\[ \Forall :{b \in B}{\Exists p{a \in A}{(b, a) \in \ConverseRel{R}}} \]

which means $\ConverseRel{r}$ is total.

\textbf{(4)} This follows by applying (3) to $\ConverseRel{r}$ and using the involution property $\ConverseRel p{\ConverseRel{r}} = r$.
\end{proof}

\begin{remark}
The characterization via converse establishes a fundamental duality between properties of relations:
\begin{itemize}
    \item Any universal property that holds for univalent relations automatically transfers to injective relations via the converse, and vice versa. If we prove that all univalent relations satisfy some property $P$, then all injective relations satisfy the "converse version" of $P$, since $r$ injective means $\ConverseRel{r}$ univalent.
    \item Similarly, any universal property of total relations automatically transfers to surjective relations via the converse, and vice versa.
\end{itemize}
This duality principle allows us to prove half as many theorems while obtaining twice as many results. For instance, if we prove a composition property for univalent relations, we immediately obtain the corresponding result for injective relations by applying the theorem to the converse.
\end{remark}

\begin{proposition}[Equivalent definitions]
    Let $r = (A, B, R)$ be a binary relation. Then:
\begin{enumerate}
    \item $r$ is univalent $\iff \Composition{R}{\ConverseRel{R}} \subseteq \Identity{B}$
    \item $r$ is total $\iff \Composition{\ConverseRel{R}}{R} \supseteq \Identity{A}$
    \item $r$ is injective $\iff \Composition{\ConverseRel{R}}{R} \subseteq \Identity{A}$
    \item $r$ is surjective $\iff \Composition{R}{\ConverseRel{R}} \supseteq \Identity{B}$
\end{enumerate}
\end{proposition}

\begin{proof}
\textbf{(1) Univalent}: 

($\Rightarrow$) Assume $r$ is univalent. Let $(b_1, b_2) \in \Composition{R}{\ConverseRel{R}}$. Then $\Exists :p{a \in A}{(b_1, a) \in \ConverseRel{R} \tand (a, b_2) \in R}$, which means $\Exists :p{a \in A}{(a, b_1) \in R \tand (a, b_2) \in R}$. Since $r$ is univalent, we have $b_1 = b_2$. Therefore $(b_1, b_2) = (b_1, b_1) \in \Identity{B}$, so $\Composition{R}{\ConverseRel{R}} \subseteq \Identity{B}$.

($\Leftarrow$) Assume $\Composition{R}{\ConverseRel{R}} \subseteq \Identity{B}$. Suppose $(a, b_1) \in R$ and $(a, b_2) \in R$. Then $(b_1, a) \in \ConverseRel{R}$ and $(a, b_2) \in R$, so $(b_1, b_2) \in \Composition{R}{\ConverseRel{R}}$. Since $\Composition{R}{\ConverseRel{R}} \subseteq \Identity{B}$, we have $(b_1, b_2) \in \Identity{B}$, which means $b_1 = b_2$. Therefore $r$ is univalent.

\textbf{(2) Total}:

($\Rightarrow$) Assume $r$ is total. Let $a \in A$ be arbitrary. Since $r$ is total, $\Exists :p{b \in B}{(a, b) \in R}$. Then $(a, b) \in R$ and $(b, a) \in \ConverseRel{R}$, so $(a, a) \in \Composition{\ConverseRel{R}}{R}$. Since $a$ was arbitrary, $(a, a) \in \Composition{\ConverseRel{R}}{R}$ for all $a \in A$. Therefore $\Identity{A} \subseteq \Composition{\ConverseRel{R}}{R}$.

($\Leftarrow$) Assume $\Composition{\ConverseRel{R}}{R} \supseteq \Identity{A}$. Let $a \in A$ be arbitrary. Then $(a, a) \in \Identity{A} \subseteq \Composition{\ConverseRel{R}}{R}$. By definition of composition, $\Exists :p{b \in B}{(a, b) \in R \tand (b, a) \in \ConverseRel{R}}$. Since $(b, a) \in \ConverseRel{R}$ is equivalent to $(a, b) \in R$, the first condition already gives us $(a, b) \in R$ for some $b \in B$. Therefore $r$ is total.

\textbf{(3) Injective}:

($\Rightarrow$) Assume $r$ is injective. Let $(a_1, a_2) \in \Composition{\ConverseRel{R}}{R}$. Then $\Exists :p{b \in B}{(a_1, b) \in R \tand (b, a_2) \in \ConverseRel{R}}$, which means $\Exists :p{b \in B}{(a_1, b) \in R \tand (a_2, b) \in R}$. Since $r$ is injective, we have $a_1 = a_2$. Therefore $(a_1, a_2) = (a_1, a_1) \in \Identity{A}$, so $\Composition{\ConverseRel{R}}{R} \subseteq \Identity{A}$.

($\Leftarrow$) Assume $\Composition{\ConverseRel{R}}{R} \subseteq \Identity{A}$. Suppose $(a_1, b) \in R$ and $(a_2, b) \in R$. Then $(a_1, b) \in R$ and $(b, a_2) \in \ConverseRel{R}$, so $(a_1, a_2) \in \Composition{\ConverseRel{R}}{R}$. Since $\Composition{\ConverseRel{R}}{R} \subseteq \Identity{A}$, we have $(a_1, a_2) \in \Identity{A}$, which means $a_1 = a_2$. Therefore $r$ is injective.

\textbf{(4) Surjective}:

($\Rightarrow$) Assume $r$ is surjective. Let $b \in B$ be arbitrary. Since $r$ is surjective, $\Exists :p{a \in A}{(a, b) \in R}$. Then $(a, b) \in R$ and $(b, a) \in \ConverseRel{R}$, so $(b, b) \in \Composition{R}{\ConverseRel{R}}$. Since $b$ was arbitrary, $(b, b) \in \Composition{R}{\ConverseRel{R}}$ for all $b \in B$. Therefore $\Identity{B} \subseteq \Composition{R}{\ConverseRel{R}}$.

($\Leftarrow$) Assume $\Composition{R}{\ConverseRel{R}} \supseteq \Identity{B}$. Let $b \in B$ be arbitrary. Then $(b, b) \in \Identity{B} \subseteq \Composition{R}{\ConverseRel{R}}$. By definition of composition, $\Exists :p{a \in A}{(b, a) \in \ConverseRel{R} \tand (a, b) \in R}$. Since $(b, a) \in \ConverseRel{R}$ is equivalent to $(a, b) \in R$, the second condition already gives us $(a, b) \in R$ for some $a \in A$. Therefore $r$ is surjective.
\end{proof}

\begin{remark}
Note that the proofs of (3) and (4) follow the same structure as (1) and (2) respectively, which is expected given the duality established by the characterization via converse: $r$ is injective if and only if $\ConverseRel{r}$ is univalent, and $r$ is surjective if and only if $\ConverseRel{r}$ is total.
\end{remark}

\begin{proposition}[Alternate Equivalent Conditions for Univalence, Totality, Injectivity, and Surjectivity]
Let $r = (A, B, R)$ be a binary relation. Let $\{s_i\}_{i \in I}$ be a family of relations and $\{B_i\}_{i \in I}$ be a family of sets, where $|I| \geq 2$ when specified. Then:
\begin{enumerate}
    \item \textbf{Total}: $r$ is total $\iff \InverseImage{r}{B} = A$.
    \item \textbf{Surjective}: $r$ is surjective $\iff \DirectImage{r}{A} = B$.
    \item \textbf{Univalent}: The following are equivalent:
        \begin{itemize}
            \item $r$ is univalent.
            \item $\Composition{(\bigcap_{i \in I} s_i)}{r} = \bigcap_{i \in I} \Composition{s_i}{r}$ for any family of relations $\{s_i\}_{i \in I}$ with $|I| \geq 2$.
            \item $\InverseImage{r}{\bigcap_{i \in I} B_i} = \bigcap_{i \in I} \InverseImage{r}{B_i}$ for any family of sets $\{B_i\}_{i \in I}$ with $|I| \geq 2$.
            \item $\InverseImage{r}{B_1 \cap B_2} = \InverseImage{r}{B_1} \cap \InverseImage{r}{B_2}$ for any sets $B_1, B_2$.
        \end{itemize}
    \item \textbf{Injective}: The following are equivalent:
        \begin{itemize}
            \item $r$ is injective.
            \item $\Composition{r}{(\bigcap_{i \in I} s_i)} = \bigcap_{i \in I} \Composition{r}{s_i}$ for any family of relations $\{s_i\}_{i \in I}$ with $|I| \geq 2$.
            \item $\DirectImage{r}{\bigcap_{i \in I} A_i} = \bigcap_{i \in I} \DirectImage{r}{A_i}$ for any family of sets $\{A_i\}_{i \in I}$ with $|I| \geq 2$.
            \item $\DirectImage{r}{A_1 \cap A_2} = \DirectImage{r}{A_1} \cap \DirectImage{r}{A_2}$ for any sets $A_1, A_2$.
        \end{itemize}
\end{enumerate}
\end{proposition}

\begin{proof}
\textbf{(1) Total}:

($\Rightarrow$) Assume $r$ is total. By definition, this means $\Domain{R} = A$. We know that $\InverseImage{r}{B} = \Domain{\Composition{\Identity{B}}{R}}$. Since $R \subseteq A \times B$, the range of $R$ is a subset of $B$, so $\Composition{\Identity{B}}{R} = R$. Thus $\InverseImage{r}{B} = \Domain{R} = A$.

($\Leftarrow$) Assume $\InverseImage{r}{B} = A$. By definition of inverse image, $\InverseImage{r}{B} = \{ x \in A \mid \Exists :{y \in B}{(x, y) \in R} \} = \Domain{R}$. Since $\Domain{R} = A$, $r$ is total.

\textbf{(2) Surjective}:

This follows from (1) by duality. $r$ is surjective $\iff \ConverseRel{r}$ is total $\iff \InverseImage{\ConverseRel{r}}{A} = B \iff \DirectImage{r}{A} = B$.

\textbf{(3) Univalent}:

We prove the equivalences in a cycle: univalent $\implies$ composition $\implies$ domain-intersection $\implies$ arbitrary intersection $\implies$ binary intersection $\implies$ univalent.

\textit{(Univalent $\implies$ Composition):} Assume $r$ is univalent. We always have $\Composition{(\bigcap_{i \in I} s_i)}{r} \subseteq \bigcap_{i \in I} \Composition{s_i}{r}$. For the reverse inclusion, let $(x, z) \in \bigcap_{i \in I} \Composition{s_i}{r}$. Then $\Forall :{i \in I}{(x, z) \in \Composition{s_i}{r}}$, so $\Forall :{i \in I}{\Exists :p{y}{(x, y) \in R \tand (y, z) \in S_i}}$. 

Since $r$ is univalent, for a given $x$, the $y$ such that $(x, y) \in R$ is unique (if it exists). Let this unique element be $y$. Then we must have $(y, z) \in S_i$ for all $i \in I$, so $(y, z) \in \bigcap_{i \in I} S_i$. Thus $(x, z) \in \Composition{(\bigcap_{i \in I} s_i)}{r}$.

\textit{(Composition $\implies$ Domain-Intersection):} Assume $\Composition{(\bigcap_{i \in I} s_i)}{r} = \bigcap_{i \in I} \Composition{s_i}{r}$ for any family of relations. We show that $\Domain{\bigcap_{i \in I} \Composition{s_i}{r}} = \bigcap_{i \in I} \Domain{\Composition{s_i}{r}}$.

We always have $\Domain{\bigcap_{i \in I} \Composition{s_i}{r}} \subseteq \bigcap_{i \in I} \Domain{\Composition{s_i}{r}}$. For the reverse inclusion, let $x \in \bigcap_{i \in I} \Domain{\Composition{s_i}{r}}$. Then $\Forall :{i \in I}{x \in \Domain{\Composition{s_i}{r}}}$, so $\Forall :{i \in I}{\Exists :p{z_i}{(x, z_i) \in \Composition{s_i}{r}}}$. By the composition assumption, $\Composition{(\bigcap_{i \in I} s_i)}{r} = \bigcap_{i \in I} \Composition{s_i}{r}$. Taking domains: $\Domain{\Composition{(\bigcap_{i \in I} s_i)}{r}} = \Domain{\bigcap_{i \in I} \Composition{s_i}{r}}$. Since $x \in \bigcap_{i \in I} \Domain{\Composition{s_i}{r}}$ and the $\subseteq$ direction holds, we need to show $x \in \Domain{\bigcap_{i \in I} \Composition{s_i}{r}}$. From $\Forall :{i \in I}{\Exists :p{z_i}{(x, z_i) \in \Composition{s_i}{r}}}$ and univalence of $r$ (which follows from the composition property), all $z_i$ must be equal. Let $z$ be this common value. Then $(x, z) \in \Composition{s_i}{r}$ for all $i$, so $(x, z) \in \bigcap_{i \in I} \Composition{s_i}{r}$, thus $x \in \Domain{\bigcap_{i \in I} \Composition{s_i}{r}}$.

\textit{(Domain-Intersection $\implies$ Arbitrary Intersection):} Assume $\Domain{\bigcap_{i \in I} \Composition{s_i}{r}} = \bigcap_{i \in I} \Domain{\Composition{s_i}{r}}$ for any family of relations. Taking $s_i = \Identity{B_i}$:
\begin{align*}
\InverseImage{r}{\bigcap_{i \in I} B_i} &= \Domain{\Composition{\Identity{\bigcap_{i \in I} B_i}}{R}} \\
&= \Domain{\Composition{(\bigcap_{i \in I} \Identity{B_i})}{r}} \\
&= \Domain{\bigcap_{i \in I} \Composition{\Identity{B_i}}{r}} \\
&= \bigcap_{i \in I} \Domain{\Composition{\Identity{B_i}}{r}} \\
&= \bigcap_{i \in I} \InverseImage{r}{B_i}
\end{align*}

\textit{(Arbitrary Intersection $\implies$ Binary Intersection):} Trivial, take $I = \{1, 2\}$.

\textit{(Binary Intersection $\implies$ Univalent):} Assume $\InverseImage{r}{B_1 \cap B_2} = \InverseImage{r}{B_1} \cap \InverseImage{r}{B_2}$ for all $B_1, B_2$. Suppose $r$ is not univalent. Then $\Exists :p{x, y_1, y_2}{(x, y_1) \in R \tand (x, y_2) \in R \tand y_1 \neq y_2}$. 

Let $B_1 = \{y_1\}$ and $B_2 = \{y_2\}$. Then $B_1 \cap B_2 = \emptyset$, so $\InverseImage{r}{B_1 \cap B_2} = \emptyset$. But $x \in \InverseImage{r}{B_1}$ and $x \in \InverseImage{r}{B_2}$, so $x \in \InverseImage{r}{B_1} \cap \InverseImage{r}{B_2}$. Contradiction.

\textbf{(4) Injective}:

This follows from (3) by duality. $r$ is injective $\iff \ConverseRel{r}$ is univalent. Applying (3) to $\ConverseRel{r}$:
\begin{itemize}
    \item $\ConverseRel{r}$ is univalent $\iff \Composition{(\bigcap_{i \in I} s_i)}{\ConverseRel{r}} = \bigcap_{i \in I} \Composition{s_i}{\ConverseRel{r}}$. Taking converses: $\Composition{r}{(\bigcap_{i \in I} \ConverseRel{s_i})} = \bigcap_{i \in I} \Composition{r}{\ConverseRel{s_i}}$, which is equivalent to the composition statement for injective relations.
    \item $\ConverseRel{r}$ is univalent $\iff \InverseImage{\ConverseRel{r}}{\bigcap_{i \in I} B_i} = \bigcap_{i \in I} \InverseImage{\ConverseRel{r}}{B_i}$. Since $\InverseImage{\ConverseRel{r}}{B} = \DirectImage{r}{B}$, this gives $\DirectImage{r}{\bigcap_{i \in I} A_i} = \bigcap_{i \in I} \DirectImage{r}{A_i}$.
    \item Similarly for the binary intersection case.
\end{itemize}
\end{proof}


\begin{proposition}[Composition Properties]
Let $r = (A, B, R)$ and $s = (B, C, S)$ be binary relations. Then:
\begin{enumerate}
    \item If $r$ and $s$ are both univalent, then $\Composition{s}{r}$ is univalent
    \item If $r$ and $s$ are both total, then $\Composition{s}{r}$ is total
    \item If $r$ and $s$ are both injective, then $\Composition{s}{r}$ is injective
    \item If $r$ and $s$ are both surjective, then $\Composition{s}{r}$ is surjective
    \item If $r$ and $s$ are both functions, then $\Composition{s}{r}$ is a function
\end{enumerate}
\end{proposition}

\begin{proof}
\textbf{(1) Univalent}: Assume $r$ and $s$ are both univalent. We need to show $\Composition{s}{r}$ is univalent, i.e., $\Composition{(\Composition{S}{R})}{\ConverseRel p{\Composition{S}{R}}} \subseteq \Identity{C}$.

By the converse of composition property, $\ConverseRel p{\Composition{S}{R}} = \Composition{\ConverseRel{R}}{\ConverseRel{S}}$. Therefore:
\begin{align*}
\Composition{(\Composition{S}{R})}{\ConverseRel p{\Composition{S}{R}}} 
&= \Composition{(\Composition{S}{R})}{(\Composition{\ConverseRel{R}}{\ConverseRel{S}})} \\
&= \Composition{S}{(\Composition{R}{\ConverseRel{R}})}{\ConverseRel{S}} \quad \text{(by associativity)} \\
&\subseteq \Composition{S}{(\Identity{B})}{\ConverseRel{S}} \quad \text{(since $r$ is univalent)} \\
&= \Composition{S}{\ConverseRel{S}} \\
&\subseteq \Identity{C} \quad \text{(since $s$ is univalent)}
\end{align*}
Therefore $\Composition{s}{r}$ is univalent.

\textbf{(2) Total}: Assume $r$ and $s$ are both total. We need to show $\Composition{s}{r}$ is total, i.e., $\Composition{\ConverseRel p{\Composition{S}{R}}}{\Composition{S}{R}} \supseteq \Identity{A}$.

By the converse of composition property:
\begin{align*}
\Composition{\ConverseRel p{\Composition{S}{R}}}{\Composition{S}{R}} 
&= \Composition{(\Composition{\ConverseRel{R}}{\ConverseRel{S}})}{(\Composition{S}{R})} \\
&= \Composition{\ConverseRel{R}}{(\Composition{\ConverseRel{S}}{\Composition{S}{R}})} \quad \text{(by associativity)} \\
&\supseteq \Composition{\ConverseRel{R}}{(\Identity{B})}{R} \quad \text{(since $s$ is total)} \\
&= \Composition{\ConverseRel{R}}{R} \\
&\supseteq \Identity{A} \quad \text{(since $r$ is total)}
\end{align*}
Therefore $\Composition{s}{r}$ is total.

\textbf{(3) Injective}: Assume $r$ and $s$ are both injective. We need to show $\Composition{s}{r}$ is injective, i.e., $\Composition{\ConverseRel p{\Composition{S}{R}}}{\Composition{S}{R}} \subseteq \Identity{A}$.

By the converse of composition property:
\begin{align*}
\Composition{\ConverseRel p{\Composition{S}{R}}}{\Composition{S}{R}} 
&= \Composition{(\Composition{\ConverseRel{R}}{\ConverseRel{S}})}{(\Composition{S}{R})} \\
&= \Composition{\ConverseRel{R}}{(\Composition{\ConverseRel{S}}{\Composition{S}{R}})} \quad \text{(by associativity)} \\
&\subseteq \Composition{\ConverseRel{R}}{(\Identity{B})}{R} \quad \text{(since $s$ is injective)} \\
&= \Composition{\ConverseRel{R}}{R} \\
&\subseteq \Identity{A} \quad \text{(since $r$ is injective)}
\end{align*}
Therefore $\Composition{s}{r}$ is injective.

\textbf{(4) Surjective}: Assume $r$ and $s$ are both surjective. We need to show $\Composition{s}{r}$ is surjective, i.e., $\Composition{(\Composition{S}{R})}{\ConverseRel p{\Composition{S}{R}}} \supseteq \Identity{C}$.

By the converse of composition property:
\begin{align*}
\Composition{(\Composition{S}{R})}{\ConverseRel p{\Composition{S}{R}}} 
&= \Composition{(\Composition{S}{R})}{(\Composition{\ConverseRel{R}}{\ConverseRel{S}})} \\
&= \Composition{S}{(\Composition{R}{\ConverseRel{R}})}{\ConverseRel{S}} \quad \text{(by associativity)} \\
&\supseteq \Composition{S}{(\Identity{B})}{\ConverseRel{S}} \quad \text{(since $r$ is surjective)} \\
&= \Composition{S}{\ConverseRel{S}} \\
&\supseteq \Identity{C} \quad \text{(since $s$ is surjective)}
\end{align*}
Therefore $\Composition{s}{r}$ is surjective.

\textbf{(5) Functions}: This follows immediately from (1) and (2), since a function is a relation that is both univalent and total.
\end{proof}

\begin{remark}
These composition properties show that the classes of total, univalent, injective, surjective, and functions relations are all closed under composition. In particular:
\begin{itemize}
    \item The composition of functions is a function
    \item The composition of injective functions is injective
    \item The composition of surjective functions is surjective
    \item The composition of bijections is a bijection
\end{itemize}
\end{remark}

\begin{proposition}[Necessary Conditions for Composition Properties]
Let $r = (A, B, R)$ and $s = (B, C, S)$ be binary relations. Then:
\begin{enumerate}
    \item \textbf{Totality}: If $\Composition{s}{r}$ is total, then $r$ is total.
    \item \textbf{Surjectivity}: If $\Composition{s}{r}$ is surjective, then $s$ is surjective.
    \item \textbf{Univalence}: If $\Composition{s}{r}$ is univalent, then $\Restriction{s}[\Range{R}]$ is univalent.
    \item \textbf{Injectivity}: If $\Composition{s}{r}$ is injective, then $\Restriction{r}[][\Domain{S}]$ is injective.
\end{enumerate}
\end{proposition}

\begin{proof}
\textbf{(1) Totality}:

Assume $\Composition{s}{r}$ is total. Let $a \in A$ be arbitrary. Since $\Composition{s}{r}$ is total, $\Exists :{c \in C}{(a, c) \in \Composition{S}{R}}$. By definition of composition, $\Exists :{b \in B}{(a, b) \in R \tand (b, c) \in S}$. Thus $\Exists :{b \in B}{(a, b) \in R}$, so $r$ is total.

\textbf{(2) Surjectivity}:

This follows from (1) by duality. $\Composition{s}{r}$ is surjective $\iff \ConverseRel p{\Composition{s}{r}}$ is total $\iff \Composition{\ConverseRel{r}}{\ConverseRel{s}}$ is total $\implies \ConverseRel{s}$ is total $\iff s$ is surjective.

\textbf{(3) Univalence}:

Assume $\Composition{s}{r}$ is univalent. We need to show $\Restriction{s}[\Range{R}]$ is univalent. Let $b_1, b_2 \in \Range{R}$ and $c \in C$ such that $(b_1, c) \in S$ and $(b_2, c) \in S$. Since $b_1, b_2 \in \Range{R}$, there exist $a_1, a_2 \in A$ such that $(a_1, b_1) \in R$ and $(a_2, b_2) \in R$. 

Then $(a_1, c) \in \Composition{S}{R}$ (via $b_1$) and $(a_2, c) \in \Composition{S}{R}$ (via $b_2$). Since $\Composition{s}{r}$ is univalent, we have $a_1 = a_2$. Let $a = a_1 = a_2$. Then $(a, b_1) \in R$ and $(a, b_2) \in R$.

Now, if $r$ were univalent, we would have $b_1 = b_2$ immediately. However, we cannot assume $r$ is univalent. Instead, we observe that for any $b \in \Range{R}$ and $c \in C$, if $(b, c) \in S$, then for any $a$ with $(a, b) \in R$, we have $(a, c) \in \Composition{S}{R}$. Since $\Composition{s}{r}$ is univalent, this $c$ is uniquely determined by $a$. But we need to show uniqueness in $b$ for fixed $c$.

Actually, let's reconsider. Suppose $(b_1, c), (b_2, c) \in S$ with $b_1, b_2 \in \Range{R}$. Pick any $a_1$ with $(a_1, b_1) \in R$ and any $a_2$ with $(a_2, b_2) \in R$. If $a_1 = a_2 = a$, then $(a, c) \in \Composition{S}{R}$ via both $b_1$ and $b_2$. Since composition is well-defined, this is fine. But if $a_1 \neq a_2$, then both $(a_1, c)$ and $(a_2, c)$ are in $\Composition{S}{R}$, which contradicts univalence of $\Composition{s}{r}$ only if they're related to the same element in $A$, which they're not.

The correct statement is: if $(b, c_1), (b, c_2) \in S$ with $b \in \Range{R}$, then $c_1 = c_2$. Let $a$ be such that $(a, b) \in R$. Then $(a, c_1), (a, c_2) \in \Composition{S}{R}$. Since $\Composition{s}{r}$ is univalent, $c_1 = c_2$. Therefore $\Restriction{s}[\Range{R}]$ is univalent.

\textbf{(4) Injectivity}:

This follows from (3) by duality. $\Composition{s}{r}$ is injective $\iff \ConverseRel p{\Composition{s}{r}}$ is univalent $\iff \Composition{\ConverseRel{r}}{\ConverseRel{s}}$ is univalent $\implies \Restriction{\ConverseRel{r}}[\Range{\ConverseRel{S}}]$ is univalent. Since $\Range{\ConverseRel{S}} = \Domain{S}$ and $\Restriction{\ConverseRel{r}}[\Domain{S}] = \ConverseRel p{\Restriction{r}[][\Domain{S}]}$, we have $\ConverseRel p{\Restriction{r}[][\Domain{S}]}$ is univalent, which means $\Restriction{r}[][\Domain{S}]$ is injective.
\end{proof}

\begin{proposition}[Characterization of Isomorphisms in Rel]
Let $r = (A, B, R)$ be a binary relation. Then $r$ is an isomorphism in the category $\mathbf{Rel}$ (i.e., there exists $s = (B, A, S)$ such that $\Composition{r}{s} = \identity{B}$ and $\Composition{s}{r} = \identity{A}$) if and only if $r$ is total, univalent, injective, and surjective.
\end{proposition}

\begin{proof}
($\Rightarrow$) Assume $r$ is an isomorphism with inverse $s = (B, A, S)$. Then $\Composition{r}{s} = \identity{B}$ and $\Composition{s}{r} = \identity{A}$.

Since $\identity{A}$ is total and injective, and $\identity{B}$ is surjective and univalent, we apply the Necessary Conditions proposition to both compositions:

\textbf{From $\Composition{s}{r} = \identity{A}$}:
\begin{itemize}
    \item $r$ is total (item 1).
    \item $s$ is surjective (item 2), so $\Range{S} = A$.
    \item $\Restriction{r}[][\Domain{S}]$ is injective (item 4).
\end{itemize}

\textbf{From $\Composition{r}{s} = \identity{B}$}:
\begin{itemize}
    \item $r$ is surjective (item 2).
    \item $s$ is total (item 1), so $\Domain{S} = B$.
    \item $\Restriction{r}[\Range{S}]$ is univalent (item 3).
\end{itemize}

Since $\Domain{S} = B$ and $\Range{S} = A$, we have $\Restriction{r}[][B] = r$ is injective and $\Restriction{r}[A] = r$ is univalent. Thus $r$ has all four properties.

($\Leftarrow$) Assume $r$ is total, univalent, injective, and surjective. We show that $s = \ConverseRel{r}$ is the inverse of $r$.

By the Equivalent definitions proposition:
\begin{align*}
\Composition{R}{\ConverseRel{R}} &\supseteq \Identity{B} \quad \text{(since $r$ is surjective)} \\
\Composition{R}{\ConverseRel{R}} &\subseteq \Identity{B} \quad \text{(since $r$ is univalent)}
\end{align*}
Therefore $\Composition{r}{\ConverseRel{r}} = \identity{B}$.

Similarly:
\begin{align*}
\Composition{\ConverseRel{R}}{R} &\supseteq \Identity{A} \quad \text{(since $r$ is total)} \\
\Composition{\ConverseRel{R}}{R} &\subseteq \Identity{A} \quad \text{(since $r$ is injective)}
\end{align*}
Therefore $\Composition{\ConverseRel{r}}{r} = \identity{A}$.

Thus $r$ is an isomorphism with inverse $\ConverseRel{r}$.
\end{proof}

\begin{remark}
This proposition shows that in the category $\mathbf{Rel}$, isomorphisms are precisely the bijective functions. The forward direction demonstrates the power of the Necessary Conditions proposition: by analyzing the composition with the inverse, we can deduce all four properties. The backward direction shows that the converse relation provides the inverse when all four properties hold.
\end{remark}


\section{Induced Functions on Power Sets}

Every binary relation $r = (A, B, R)$ induces two functions between the power sets of its domain and codomain.

\begin{proposition}[Induced Functions from Relations]
Let $r = (A, B, R)$ be a binary relation. Then $r$ induces:
\begin{enumerate}
    \item A function $r_* : \mathcal{P}(A) \to \mathcal{P}(B)$ defined by $r_*(N) = \DirectImage{r}{N}$ for all $N \subseteq A$.
    \item A function $r^* : \mathcal{P}(B) \to \mathcal{P}(A)$ defined by $r^*(M) = \InverseImage{r}{M}$ for all $M \subseteq B$.
\end{enumerate}
\end{proposition}

\begin{proof}
Both are well-defined functions since for any subset of the domain/codomain, the direct/inverse image is uniquely determined and is a subset of the codomain/domain respectively.
\end{proof}

\begin{remark}
Not all functions from $\mathcal{P}(A)$ to $\mathcal{P}(B)$ arise from binary relations. We now characterize precisely which functions can be induced by relations.
\end{remark}

\begin{example}[Functions Not Induced by Relations]
Let $A = \{1, 2\}$ and $B = \{a, b\}$. Consider the function $f : \mathcal{P}(A) \to \mathcal{P}(B)$ defined by:
\begin{align*}
f(\emptyset) &= \{a\} \\
f(\{1\}) &= \{b\} \\
f(\{2\}) &= \{a\} \\
f(\{1, 2\}) &= \{a, b\}
\end{align*}

This function cannot be induced by any relation $r = (A, B, R)$ because $f(\emptyset) = \{a\} \neq \emptyset$, but for any relation, $\DirectImage{r}{\emptyset} = \emptyset$.
\end{example}

\begin{proposition}[Characterization of Functions Induced by Relations]
A function $f : \mathcal{P}(A) \to \mathcal{P}(B)$ is induced by some binary relation $r = (A, B, R)$ (i.e., $f = r_*$) if and only if $f$ distributes over arbitrary unions:
\[ f(\bigcup_{i \in I} N_i) = \bigcup_{i \in I} f(N_i) \]
for any family $\{N_i\}_{i \in I}$ of subsets of $A$.
\end{proposition}

\begin{proof}
($\Rightarrow$) Assume $f = r_*$ for some relation $r = (A, B, R)$.

By the properties of direct images established earlier, $\DirectImage{r}{\bigcup_{i \in I} N_i} = \bigcup_{i \in I} \DirectImage{r}{N_i}$.

($\Leftarrow$) Assume $f$ distributes over arbitrary unions. Define $R = \{(a, b) \in A \times B \mid b \in f(\{a\})\}$. We claim that $f = r_*$ where $r = (A, B, R)$.

For any $N \subseteq A$, we can write $N = \bigcup_{a \in N} \{a\}$. Then:
\begin{align*}
\DirectImage{r}{N} &= \{b \in B \mid \Exists :{a \in N}{(a, b) \in R}\} \\
&= \{b \in B \mid \Exists :{a \in N}{b \in f(\{a\})}\} \\
&= \bigcup_{a \in N} f(\{a\}) \\
&= f\left(\bigcup_{a \in N} \{a\}\right) \quad \text{(by distributivity over unions)} \\
&= f(N)
\end{align*}

Thus $f = r_*$.
\end{proof}

\begin{remark}
Note that the property $f(\emptyset) = \emptyset$ is automatically satisfied since taking $I = \emptyset$ in the arbitrary union property gives $f\left(\bigcup_{i \in \emptyset} N_i\right) = f(\emptyset) = \bigcup_{i \in \emptyset} f(N_i) = \emptyset$.
\end{remark}






\chapter{Homogeneous Relations}

\section{Definition}

A binary relation $r = (A, A, R)$ where the domain and codomain are the same set is called a \textbf{homogeneous relation} or \textbf{endorelation} on $A$.

Homogeneous relations have special properties that do not apply to general heterogeneous relations. In this section, we explore these properties.

\section{Reflexivity, Symmetry, and Transitivity}

\begin{definition}[Reflexive Relation]
A homogeneous relation $r = (A, A, R)$ is \textbf{reflexive} if every element is related to itself:
\[ \Forall :{a \in A}{(a, a) \in R} \]
Equivalently, $\Identity{A} \subseteq R$.
\end{definition}

\begin{definition}[Irreflexive Relation]
A homogeneous relation $r = (A, A, R)$ is \textbf{irreflexive} if no element is related to itself:
\[ \Forall :{a \in A}{(a, a) \notin R} \]
Equivalently, $R \cap \Identity{A} = \emptyset$.
\end{definition}

\begin{definition}[Symmetric Relation]
A homogeneous relation $r = (A, A, R)$ is \textbf{symmetric} if whenever $a$ is related to $b$, then $b$ is related to $a$:
\[ \Forall :{a, b \in A}{(a, b) \in R \implies (b, a) \in R} \]
Equivalently, $R \subseteq \ConverseRel{R}$ in which case $R = \ConverseRel{R}$.
\end{definition}

\begin{definition}[Asymmetric Relation]
A homogeneous relation $r = (A, A, R)$ is \textbf{asymmetric} if whenever $a$ is related to $b$, then $b$ is not related to $a$:
\[ \Forall :{a, b \in A}{(a, b) \in R \implies (b, a) \notin R} \]
Equivalently, $R \cap \ConverseRel{R} = \emptyset$.
\end{definition}

\begin{definition}[Antisymmetric Relation]
A homogeneous relation $r = (A, A, R)$ is \textbf{antisymmetric} if whenever $a$ is related to $b$ and $b$ is related to $a$, then $a = b$:
\[ \Forall :{a, b \in A}{(a, b) \in R \tand (b, a) \in R \implies a = b} \]
Equivalently, $R \cap \ConverseRel{R} \subseteq \Identity{A}$.
\end{definition}

\begin{definition}[Transitive Relation]
A homogeneous relation $r = (A, A, R)$ is \textbf{transitive} if whenever $a$ is related to $b$ and $b$ is related to $c$, then $a$ is related to $c$:
\[ \Forall :{a, b, c \in A}{(a, b) \in R \tand (b, c) \in R \implies (a, c) \in R} \]
Equivalently, $\Composition{R}{R} \subseteq R$.
\end{definition}

\begin{definition}[Connected Relation]
A homogeneous relation $r = (A, A, R)$ is \textbf{connected} (or \textbf{complete}) if for any two distinct elements, at least one is related to the other:
\[ \Forall :{a, b \in A}{a \neq b \implies (a, b) \in R \tor (b, a) \in R} \]
Equivalently, $R \cup \ConverseRel{R} \cup \Identity{A} = A \times A$.
\end{definition}

\begin{definition}[Strongly Connected Relation]
A homogeneous relation $r = (A, A, R)$ is \textbf{strongly connected} (or \textbf{total}) if for any two elements (not necessarily distinct), at least one is related to the other:
\[ \Forall :{a, b \in A}{(a, b) \in R \tor (b, a) \in R} \]
Equivalently, $R \cup \ConverseRel{R} = A \times A$.
\end{definition}

\begin{definition}[Trichotomous Relation]
A homogeneous relation $r = (A, A, R)$ is \textbf{trichotomous} if for any two elements, exactly one of the following holds: $a$ is related to $b$, $b$ is related to $a$, or $a = b$:
\[ \Forall :{a, b \in A}{(a, b) \in R \oplus (b, a) \in R \oplus a = b} \]
where $\oplus$ denotes exclusive or. Equivalently, $R$, $\ConverseRel{R}$, and $\Identity{A}$ are pairwise disjoint and $R \cup \ConverseRel{R} \cup \Identity{A} = A \times A$.
\end{definition}

\begin{definition}[Minimal Element Property (Non-Strict)]
A homogeneous relation $r = (A, A, R)$ has the \textbf{minimal element property} if every nonempty subset of $A$ has a minimal element with respect to $R$:
\[ \Forall :{S \subseteq A}{S \neq \emptyset \implies \Exists :{m \in S}{\Forall :{s \in S}{(s, m) \in R \implies s = m}}} \]
\end{definition}

\begin{definition}[Minimal Element Property (Strict)]
A homogeneous relation $r = (A, A, R)$ has the \textbf{strict minimal element property} if every nonempty subset of $A$ has a strict minimal element with respect to $R$:
\[ \Forall :{S \subseteq A}{S \neq \emptyset \implies \Exists :{m \in S}{\Forall :{s \in S}{(s, m) \notin R}}} \]
\end{definition}

\begin{definition}[Minimum Element Property (Non-Strict)]
A homogeneous relation $r = (A, A, R)$ has the \textbf{minimum element property} if every nonempty subset of $A$ has a minimum element with respect to $R$:
\[ \Forall :{S \subseteq A}{S \neq \emptyset \implies \Exists :{m \in S}{\Forall :{s \in S}{(m, s) \in R}}} \]
\end{definition}

\begin{definition}[Minimum Element Property (Strict)]
A homogeneous relation $r = (A, A, R)$ has the \textbf{strict minimum element property} if every nonempty subset of $A$ has a strict minimum element with respect to $R$:
\[ \Forall :{S \subseteq A}{S \neq \emptyset \implies \Exists :{m \in S}{\Forall :{s \in S}{s \neq m \implies (m, s) \in R}}} \]
\end{definition}

\section{Special Types of Homogeneous Relations}

\begin{definition}[Preorder Relation]
A homogeneous relation $r = (A, A, R)$ is a \textbf{preorder relation} if it is reflexive and transitive.
\end{definition}

\begin{definition}[Equivalence Relation]
A homogeneous relation $r = (A, A, R)$ is an \textbf{equivalence relation} if it is reflexive, symmetric, and transitive.
\end{definition}

\begin{definition}[Partial Order]
A homogeneous relation $r = (A, A, R)$ is a \textbf{partial order} if it is reflexive, antisymmetric, and transitive.
\end{definition}

\begin{definition}[Strict Partial Order]
A homogeneous relation $r = (A, A, R)$ is a \textbf{strict partial order} if it is irreflexive, asymmetric, and transitive.
\end{definition}

\begin{definition}[Total Order]
A homogeneous relation $r = (A, A, R)$ is a \textbf{total order} (or \textbf{linear order}) if it is a partial order that is also connected. Equivalently, it is reflexive, antisymmetric, transitive, and connected.
\end{definition}

\begin{definition}[Strict Total Order]
A homogeneous relation $r = (A, A, R)$ is a \textbf{strict total order} if it is a strict partial order that is also trichotomous. Equivalently, it is irreflexive, asymmetric, transitive, and trichotomous.
\end{definition}

\begin{remark}
These definitions form a hierarchy of increasingly structured relations:
\begin{itemize}
    \item Equivalence relations partition sets into equivalence classes
    \item Partial orders provide a notion of comparison without requiring all elements to be comparable
    \item Total orders extend partial orders by requiring all elements to be comparable
    \item Well-orders are total orders with the additional property that every subset has a least element
\end{itemize}
\end{remark}





