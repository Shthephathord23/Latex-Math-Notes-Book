\chapter{Binary Relations}

\section{Definition and Basic Properties}

The following are equivalent for any set $R$:

\begin{itemize}
    \item $\Exists p{A}{\Exists pp{B}{R \subseteq A \times B}}$.
    \item $\Forall: {x \in R}{\Exists p{y}{\Exists pp{z}{x = (y, z)}}}$.
\end{itemize}

\begin{proof}
    \textbf{($\Rightarrow$)} Assume $\Exists p{A}{\Exists pp{B}{R \subseteq A \times B}}$.

    Let $x \in R$ be arbitrary. Since $R \subseteq A \times B$, we have $x \in A \times B$. By definition of Cartesian product, $x = (a, b)$ for some $a \in A$ and $b \in B$. 

    Taking $y = a$ and $z = b$, we have shown that $\Forall :{x \in R}{\Exists p{y}{\Exists pp{z}{x = (y, z)}}}$.

    \textbf{($\Leftarrow$)} Assume $\Forall :{x \in R}{\Exists p{y}{\Exists pp{z}{x = (y, z)}}}$.

    We need to find sets $A$ and $B$ such that $R \subseteq A \times B$.

    By the \textbf{Axiom of Replacement}, since for each $x \in R$ there exists a unique first component $y$ such that $x = (y, z)$ for some $z$, we can form the set:
    \[\Domain{R} \defeq \left\{ y \mid \Exists{x \in R}{\Exists{z}{x = (y, z)}} \right\} \]
    % called the \textbf{domain} of $R$.

    Similarly, we can form the set:
    \[\Range{R} \defeq \left\{ z \mid \Exists{x \in R}{\Exists{y}{x = (y, z)}} \right\} \]
    % called the \textbf{range} of $R$.

    More precisely, by the Axiom of Replacement applied to the function $f(x) = \text{"first component of } x \text{"}$ on the set $R$, we obtain $A$. Similarly for $B$ using the function $g(x) = \text{"second component of } x \text{"}$.

    Now let $x \in R$. By assumption, $\Exists p{y}{\Exists pp{z}{x = (y, z)}}$. By construction of $A$ and $B$ using replacement, we have $y \in A$ and $z \in B$, so $x = (y, z) \in A \times B$.

    Therefore $R \subseteq A \times B$, which proves $\Exists p{A}{\Exists pp{B}{R \subseteq A \times B}}$.
\end{proof}

\begin{remark}
    Notice that the sets $A$ and $B$ are not unique, but they must obey $\Domain{R} \subseteq A$ and $\Range{R} \subseteq B$.
\end{remark}

\begin{definition}[Binary Relations and Graphs of Binary Relations]
    A \textbf{graph of a binary relation} is a set $R$ that satisfies one of the following equivalent conditions above. \\
    A \textbf{binary relation} is a tuple $(A, B, R)$ such that $R \subseteq A \times B$. \\
    We will denote graphs of binary relations with capital letters, such as $R, S$ etc. and binary relations with lowercase letters, such as $r, s$ etc.
\end{definition}

\begin{definition}[Domain and Range for a Graph of a Binary Relation]
    Let $R \subseteq A \times B$ be a graph of a binary relation. The \textbf{domain} of $R$, denoted $\Domain{R}$, is the set:
    \[\Domain{R} \defeq \left\{ x \in A \mid \Exists{y \in B}{(x, y) \in R} \right\}\]
    The \textbf{range} of $R$, denoted $\Range{R}$, is the set:
    \[\Range{R} \defeq \left\{ y \in B \mid \Exists{x \in A}{(x, y) \in R} \right\}\]
\end{definition}


\subsection{Examples of Binary Relations}

\begin{example}[Identity Relation]
Let $A$ be any set. The \textbf{identity graph} on $A$ is defined as:
\[\text{Id}_A \defeq \left\{ (a, a) \mid a \in A \right\}\]
This is the identity relation $\text{id}_A \defeq (A, A, \text{Id}_A)$ where $\text{Id}_A \subseteq A \times A$.
\end{example}

\begin{example}[Empty Relation]
Let $A$ and $B$ be any sets. The \textbf{empty relation} from $A$ to $B$ is:
\[\emptyset \subseteq A \times B\]
This gives us the binary relation $(A, B, \emptyset)$.
\end{example}

\begin{example}[Universal Relation]
Let $A$ and $B$ be any sets. The \textbf{universal relation} from $A$ to $B$ is:
\[\UniversalRel{A}[B] \defeq A \times B\]
This gives us the binary relation $(A, B, \UniversalRel{A}[B])$ where every element of $A$ is related to every element of $B$.
\end{example}

\begin{remark}
These three examples represent the extreme cases:
\begin{itemize}
    \item The identity relation relates each element only to itself
    \item The empty relation relates no elements
    \item The universal relation relates every possible pair of elements
\end{itemize}
\end{remark}

\subsection{The Complement of a Binary Relation}
\begin{definition}[Complement of a binary relation]
Let $r = (A, B, R)$ be a binary relation. The \textbf{complement} of $r$, denoted $\ComplementRel{r}$, is the binary relation $(A, B, \ComplementRel{R})$ where:
\[\ComplementRel{R} \defeq \UniversalRel{A}[B] \setminus R = \left\{ (x, y) \in A \times B \mid (x, y) \notin R \right\}\]
\end{definition}

\begin{remark}
Note that the complement operation is only well-defined for binary relations (tuples with specified domain and codomain), not for arbitrary graphs of binary relations, since we need the universal set $A \times B$ to form the complement.
\end{remark}

\begin{proposition}[Properties of Complement for Binary Relations]
Let $r = (A, B, R)$ and $s = (A, B, S)$ be binary relations with the same domain and codomain. Then:
\begin{enumerate}
    \item \textbf{Involution}: $\ComplementRel{(\ComplementRel{r})} = r$
    \item \textbf{De Morgan's Laws}: $\ComplementRel{(r \cup s)} = \ComplementRel{r} \cap \ComplementRel{s}$ and $\ComplementRel{(r \cap s)} = \ComplementRel{r} \cup \ComplementRel{s}$
    \item \textbf{Antimonotonicity}: If $R \subseteq S$ then $\ComplementRel{S} \subseteq \ComplementRel{R}$
    \item \textbf{Universal and Empty}: The universal relation $(A, B, \UniversalRel{A}[B])$ has complement $(A, B, \emptyset)$, and vice versa
\end{enumerate}
\end{proposition}

\begin{proof}
\textbf{(1) Involution}: 
\begin{align}
\ComplementRel{(\ComplementRel{r})} &= (A, B, \ComplementRel{((A \times B) \setminus R)}) \\
&= (A, B, (A \times B) \setminus ((A \times B) \setminus R)) \\
&= (A, B, R) = r
\end{align}

\textbf{(2) De Morgan's Laws}: For the first law:
\begin{align}
\ComplementRel{(r \cup s)} &= (A, B, \ComplementRel{(R \cup S)}) \\
&= (A, B, (A \times B) \setminus (R \cup S)) \\
&= (A, B, ((A \times B) \setminus R) \cap ((A \times B) \setminus S)) \\
&= (A, B, \ComplementRel{R} \cap \ComplementRel{S}) = \ComplementRel{r} \cap \ComplementRel{s}
\end{align}
The second law follows similarly.

\textbf{(3) Antimonotonicity}: Assume $R \subseteq S$. Let $(x,y) \in \ComplementRel{S}$. Then $(x,y) \in A \times B$ and $(x,y) \notin S$. Since $R \subseteq S$, if $(x,y)$ were in $R$ then it would be in $S$, which is not the case. Therefore $(x,y) \notin R$, so $(x,y) \in \ComplementRel{R}$.

\textbf{(4) Universal and Empty}: Direct from the definition of complement.
\end{proof}

\subsection{Operations on Binary Relations}

Since binary relations are defined as tuples $(A, B, R)$ where $R \subseteq A \times B$, we can define set-theoretic operations on them when they share the same domain and codomain.

\begin{definition}[Arbitrary Union of Binary Relations]
Let $\{r_i\}_{i \in I} = \{(A, B, R_i)\}_{i \in I}$ be a family of binary relations with the same domain $A$ and codomain $B$. Their \textbf{union}, denoted $\bigcup_{i \in I} r_i$, is defined as:
\[ \bigcup_{i \in I} r_i \defeq \left( A, B, \bigcup_{i \in I} R_i \right) \]
\end{definition}

\begin{definition}[Arbitrary Intersection of Binary Relations]
Let $\{r_i\}_{i \in I} = \{(A, B, R_i)\}_{i \in I}$ be a family of binary relations with the same domain $A$ and codomain $B$. Their \textbf{intersection}, denoted $\bigcap_{i \in I} r_i$, is defined as:
\[ \bigcap_{i \in I} r_i \defeq \left( A, B, \bigcap_{i \in I} R_i \right) \]
\end{definition}

\begin{definition}[Subset Relation for Binary Relations]
Let $r = (A, B, R)$ and $s = (A, B, S)$ be binary relations with the same domain $A$ and codomain $B$. We say $r \subseteq s$ if and only if $R \subseteq S$.
\end{definition}

\subsection{Converse of a Binary Relation}
\begin{definition}[Converse of the graph of a binary relation]
Let $R$ be a graph of a binary relation. The \textbf{converse} (or \textbf{transpose}) of $R$, denoted $\ConverseRel{R}$ or $R^{-1}$, is defined as:
\[\ConverseRel{R} \defeq \left\{ (y, x) \mid (x, y) \in R \right\}\]
\end{definition}

\begin{definition}[Converse of a binary relation]
Let $r = (A, B, R)$ be a binary relation. The \textbf{converse} of $r$, denoted $\ConverseRel{r}$, is the binary relation $(B, A, \ConverseRel{R})$.
\end{definition}

\begin{proposition}[Properties of Converse]
Let $R$ and $S$ be graphs of binary relations. Then:
\begin{enumerate}
    \item \textbf{Involution}: $\ConverseRel{(\ConverseRel{R})} = R$
    \item \textbf{Domain and Range}: $\Domain{\ConverseRel{R}} = \Range{R}$ and $\Range{\ConverseRel{R}} = \Domain{R}$
    \item \textbf{Union}: $\ConverseRel{(R \cup S)} = \ConverseRel{R} \cup \ConverseRel{S}$
    \item \textbf{Intersection}: $\ConverseRel{(R \cap S)} = \ConverseRel{R} \cap \ConverseRel{S}$
    \item \textbf{Monotonicity}: If $R \subseteq S$ then $\ConverseRel{R} \subseteq \ConverseRel{S}$
    \item \textbf{Identity}: $\ConverseRel{(\text{Id}_A)} = \text{Id}_A$ for any set $A$
    \item \textbf{Empty and Universal}: $\ConverseRel{\emptyset} = \emptyset$ and $\ConverseRel{(\UniversalRel{A}[B])} = \UniversalRel{B}[A]$
\end{enumerate}
\end{proposition}

\begin{proof}
\textbf{(1) Involution}: For any $(x, y)$:
\begin{align}
(x, y) \in \ConverseRel{(\ConverseRel{R})} &\Leftrightarrow (y, x) \in \ConverseRel{R} \\
&\Leftrightarrow (x, y) \in R
\end{align}
Therefore $\ConverseRel{(\ConverseRel{R})} = R$.

\textbf{(2) Domain and Range}: 
\begin{align}
\Domain{\ConverseRel{R}} &= \left\{ x \mid \Exists{y}{(x, y) \in \ConverseRel{R}} \right\} \\
&= \left\{ x \mid \Exists{y}{(y, x) \in R} \right\} \\
&= \left\{ x \mid x \in \Range{R} \right\} = \Range{R}
\end{align}
Similarly, $\Range{\ConverseRel{R}} = \Domain{R}$.

\textbf{(3) Union}: For any $(x, y)$:
\begin{align}
(x, y) \in \ConverseRel{(R \cup S)} &\Leftrightarrow (y, x) \in R \cup S \\
&\Leftrightarrow (y, x) \in R \text{ or } (y, x) \in S \\
&\Leftrightarrow (x, y) \in \ConverseRel{R} \text{ or } (x, y) \in \ConverseRel{S} \\
&\Leftrightarrow (x, y) \in \ConverseRel{R} \cup \ConverseRel{S}
\end{align}

\textbf{(4) Intersection}: For any $(x, y)$:
\begin{align}
(x, y) \in \ConverseRel{(R \cap S)} &\Leftrightarrow (y, x) \in R \cap S \\
&\Leftrightarrow (y, x) \in R \text{ and } (y, x) \in S \\
&\Leftrightarrow (x, y) \in \ConverseRel{R} \text{ and } (x, y) \in \ConverseRel{S} \\
&\Leftrightarrow (x, y) \in \ConverseRel{R} \cap \ConverseRel{S}
\end{align}

\textbf{(5) Monotonicity}: Assume $R \subseteq S$. Let $(x, y) \in \ConverseRel{R}$. Then $(y, x) \in R$. Since $R \subseteq S$, we have $(y, x) \in S$, which implies $(x, y) \in \ConverseRel{S}$. Therefore $\ConverseRel{R} \subseteq \ConverseRel{S}$.

\textbf{(6) Identity}: For any $(x, y)$:
\begin{align}
(x, y) \in \ConverseRel{(\text{Id}_A)} &\Leftrightarrow (y, x) \in \text{Id}_A \\
&\Leftrightarrow y = x \text{ and } y \in A \\
&\Leftrightarrow x = y \text{ and } x \in A \\
&\Leftrightarrow (x, y) \in \text{Id}_A
\end{align}

\textbf{(7) Empty and Universal}: 
For the empty relation: $(x, y) \in \ConverseRel{\emptyset} \Leftrightarrow (y, x) \in \emptyset \Leftrightarrow \text{false} \Leftrightarrow (x, y) \in \emptyset$.

For the universal relation: $(x, y) \in \ConverseRel{(A \times B)} \Leftrightarrow (y, x) \in A \times B \Leftrightarrow y \in A \text{ and } x \in B \Leftrightarrow (x, y) \in B \times A$.
\end{proof}

\begin{proposition}[Properties of Converse for Binary Relations]
Let $r = (A, B, R)$ and $s = (A, B, S)$ be binary relations with the same domain and codomain. Then:
\begin{enumerate}
    \item \textbf{Involution}: $\ConverseRel{(\ConverseRel{r})} = r$
    \item \textbf{Domain and Codomain}: $\Domain{\ConverseRel{r}} = \Range{r}$ and $\Codomain{\ConverseRel{r}} = \Domain{r}$
    \item \textbf{Union}: $\ConverseRel{(r \cup s)} = \ConverseRel{r} \cup \ConverseRel{s}$
    \item \textbf{Intersection}: $\ConverseRel{(r \cap s)} = \ConverseRel{r} \cap \ConverseRel{s}$
    \item \textbf{Monotonicity}: If $r \subseteq s$ then $\ConverseRel{r} \subseteq \ConverseRel{s}$
    \item \textbf{Identity}: $\ConverseRel{\text{id}_A} = \text{id}_A$ 
    \item \textbf{Empty and Universal}: $\ConverseRel{\emptyset_{A,B}} = \emptyset_{B,A}$ and $\ConverseRel{(\UniversalRel{A}[B])} = \UniversalRel{B}[A]$
\end{enumerate}
\end{proposition}

\begin{proof}
These properties follow directly from the corresponding properties for graphs of binary relations, since the converse operation is defined componentwise on the underlying graph.
\end{proof}

\begin{example}
Let $A = \left\{ 1, 2 \right\}$ and $B = \left\{ a, b \right\}$. Consider the graph of a binary relation:
\[R = \left\{ (1, a), (1, b), (2, a) \right\} \subseteq A \times B\]
Then its converse is:
\[\ConverseRel{R} = \left\{ (a, 1), (b, 1), (a, 2) \right\} \subseteq B \times A\]
Notice that $\Domain{R} = \left\{ 1, 2 \right\} = \Range{\ConverseRel{R}}$ and $\Range{R} = \left\{ a, b \right\} = \Domain{\ConverseRel{R}}$.
\end{example}

\section{Composition of Binary Relations}

\begin{definition}[Composition of graphs of binary relations]
Let $R$ and $S$ be graphs of binary relations. The \textbf{composition} of $R$ and $S$, denoted $\Composition{S}{R}$, is defined as:
\[ \Composition{S}{R} \defeq \left\{ (a, c) \mid \Exists pp{b}{(a, b) \in R \text{ and } (b, c) \in S} \right\} \]
\end{definition}

\begin{definition}[Composition of binary relations]
Let $r = (A, B, R)$ and $s = (B, C, S)$ be binary relations. The \textbf{composition} of $r$ and $s$, denoted $\Composition{s}{r}$, is the binary relation $(A, C, \Composition{S}{R})$.
\end{definition}

\begin{remark}
Note the order: $\Composition{s}{r}$ means we first apply relation $r$, then relation $s$. This follows the standard function composition convention. \\
The composition of graphs of binary relations is defined for all graphs, but the composition of binary relations is only defined when the codomain of the first relation matches the domain of the second relation.
\end{remark}

\begin{example}
Let $A = \left\{ 1, 2 \right\}$, $B = \left\{ a, b, c \right\}$, and $C = \left\{ x, y \right\}$. Consider:
\begin{align}
R &= \left\{ (1, a), (1, b), (2, c) \right\} \\
S &= \left\{ (a, x), (b, x), (c, y) \right\}
\end{align}
Then:
\[\Composition{S}{R} = \left\{ (1, x), (2, y) \right\}\]
\end{example}

\subsection{Basic Properties of Composition}

\begin{proposition}[Basic Properties of Composition for graphs]
Let $R$, $S$, and $T$ be graphs of binary relations. Then:
\begin{enumerate}
    \item $\Domain{\Composition{S}{R}} \subseteq \Domain{R}$
    \item $\Range{\Composition{S}{R}} \subseteq \Range{S}$
    \item $\Composition{S}{R} = \{(a,c) \mid \Exists{b \in \Range{R} \cap \Domain{S}}{(a,b) \in R \text{ and } (b,c) \in S}\}$
    \item \textbf{Associativity}: $\Composition{(\Composition{T}{S})}{R} = \Composition{T}{(\Composition{S}{R})}$
    \item \textbf{Identity}: $\Composition{\text{Id}_B}{R} = R$ and $\Composition{R}{\text{Id}_A} = R$ when $R \subseteq A \times B$
    \item \textbf{Converse of Composition}: $\ConverseRel{(\Composition{S}{R})} = \Composition{\ConverseRel{R}}{\ConverseRel{S}}$
\end{enumerate}
\end{proposition}

\begin{proof}
\textbf{(1)} Let $a \in \Domain{\Composition{S}{R}}$. Then $\Exists{c}{(a,c) \in \Composition{S}{R}}$, so $\Exists{b}{(a,b) \in R \text{ and } (b,c) \in S}$. Thus $a \in \Domain{R}$.

\textbf{(2)} Let $c \in \Range{\Composition{S}{R}}$. Then $\Exists{a}{(a,c) \in \Composition{S}{R}}$, so $\Exists{b}{(a,b) \in R \text{ and } (b,c) \in S}$. Thus $c \in \Range{S}$.

\textbf{(3)} By definition:
\[
\Composition{S}{R} = \left\{ (a,c) \mid \Exists{b}{(a,b) \in R \text{ and } (b,c) \in S} \right\}
\]
If $(a,b) \in R$ then $b \in \Range{R}$, and if $(b,c) \in S$ then $b \in \Domain{S}$. Thus:
\[
\Exists{b}{(a,b) \in R \text{ and } (b,c) \in S} \Leftrightarrow \Exists{b \in \Range{R} \cap \Domain{S}}{(a,b) \in R \text{ and } (b,c) \in S}
\]
Therefore the equality holds.

\textbf{(4) Associativity}: 
For any $(a,d)$:
\begin{align}
(a,d) \in \Composition{(\Composition{T}{S})}{R} 
&\Leftrightarrow \Exists b{(a,b) \in R \text{ and } (b,d) \in \Composition{T}{S}} \\
&\Leftrightarrow \Exists b{(a,b) \in R \text{ and } \Exists c{(b,c) \in S \text{ and } (c,d) \in T}} \\
&\Leftrightarrow \Exists c{\Exists b{(a,b) \in R \text{ and } (b,c) \in S} \text{ and } (c,d) \in T} \\
&\Leftrightarrow \Exists c{(a,c) \in \Composition{S}{R} \text{ and } (c,d) \in T} \\
&\Leftrightarrow (a,d) \in \Composition{T}{(\Composition{S}{R})}
\end{align}

\textbf{(5) Identity}: 
First part: Assume $R \subseteq A \times B$. For any $(a,b)$:
\begin{align}
(a,b) \in \Composition{\text{Id}_A}{R} 
&\Leftrightarrow \Exists x{(a,x) \in \text{Id}_A \text{ and } (x,b) \in R} \\
&\Leftrightarrow \Exists x \in A{a = x \text{ and } (x,b) \in R} \\
&\Leftrightarrow (a,b) \in R
\end{align}
Similarly for $\Composition{R}{\text{Id}_A} = R$.

\textbf{(6) Converse of Composition}: 
For any $(a,c)$:
\begin{align}
(a,c) \in \ConverseRel{(\Composition{S}{R})}
&\Leftrightarrow (c,a) \in \Composition{S}{R} \\
&\Leftrightarrow \Exists b{(c,b) \in R \text{ and } (b,a) \in S} \\
&\Leftrightarrow \Exists b{(b,c) \in \ConverseRel{R} \text{ and } (a,b) \in \ConverseRel{S}} \\
&\Leftrightarrow (a,c) \in \Composition{\ConverseRel{R}}{\ConverseRel{S}}
\end{align}
\end{proof}

\begin{corollary}[Basic Properties of Composition for binary relations]
Let $r$, $s$, and $t$ be binary relations where the compositions are defined. Then:
\begin{enumerate}
    \item $\Composition{(\Composition{t}{s})}{r} = \Composition{t}{(\Composition{s}{r})}$ (Associativity)
    \item \textbf{Identity}: If $\text{id}_B = (B, B, \text{Id}_B)$ is the identity relation on $B$, then:
    \begin{itemize}
        \item $\Composition{\text{id}_B}{r} = r$ for any relation $r = (A, B, R)$
        \item $\Composition{s}{\text{id}_B} = s$ for any relation $s = (B, C, S)$
    \end{itemize}
\end{enumerate}
\end{corollary}

\begin{proof}
These properties follow immediately from the corresponding properties for graphs of binary relations, since the composition of binary relations $(A, C, \Composition{S}{R})$ is defined in terms of the composition of their underlying graphs $\Composition{S}{R}$.
\end{proof}

\subsection{Additional Properties of Composition}

\begin{proposition}[Algebraic Properties of Composition for graphs]
Let $R, S, T$ be graphs of binary relations, and let $\{R_i\}_{i \in I}$ and $\{S_i\}_{i \in I}$ be families of graphs of binary relations where the compositions are defined. Then:
\begin{enumerate}
    \item \textbf{Distributivity over Union}:
    \begin{itemize}
        \item Left distributivity: $\Composition{S}{\left( \bigcup_{i \in I} R_i \right)} = \bigcup_{i \in I} \Composition{S}{R_i}$
        \item Right distributivity: $\Composition{\left( \bigcup_{i \in I} S_i \right)}{R} = \bigcup_{i \in I} \Composition{S_i}{R}$
    \end{itemize}
    
    \item \textbf{Subdistributivity over Intersection}:
    \begin{itemize}
        \item Left subdistributivity: $\Composition{S}{\left( \bigcap_{i \in I} R_i \right)} \subseteq \bigcap_{i \in I} \Composition{S}{R_i}$
        \item Right subdistributivity: $\Composition{\left( \bigcap_{i \in I} S_i \right)}{R} \subseteq \bigcap_{i \in I} \Composition{S_i}{R}$
    \end{itemize}
    
    \item \textbf{Monotonicity}:
    \begin{itemize}
        \item If $R_1 \subseteq R_2$ then $\Composition{S}{R_1} \subseteq \Composition{S}{R_2}$ (Left monotonicity)
        \item If $S_1 \subseteq S_2$ then $\Composition{S_1}{R} \subseteq \Composition{S_2}{R}$ (Right monotonicity)
    \end{itemize}
    
    \item \textbf{Antimonotonicity with Complement}:
    \begin{itemize}
        \item $\Composition{\ComplementRel {S}}{R} \subseteq \ComplementRel p{\Composition{S}{R}}$
        \item $\Composition{S}{\ComplementRel {R}} \subseteq \ComplementRel p{\Composition{S}{R}}$
    \end{itemize}
\end{enumerate}
\end{proposition}


\begin{proof}
\textbf{(1) Distributivity over Union}:

For left distributivity:
\begin{align}
(a,c) \in \Composition{S}{\left( \bigcup_{i \in I} R_i \right)} 
&\Leftrightarrow \Exists b{(a,b) \in \bigcup_{i \in I} R_i \text{ and } (b,c) \in S} \\
&\Leftrightarrow \Exists b{\left( \Exists i \in I{(a,b) \in R_i} \right) \text{ and } (b,c) \in S} \\
&\Leftrightarrow \Exists i \in I{\Exists b{(a,b) \in R_i \text{ and } (b,c) \in S}} \\
&\Leftrightarrow \Exists i \in I{(a,c) \in \Composition{S}{R_i}} \\
&\Leftrightarrow (a,c) \in \bigcup_{i \in I} \Composition{S}{R_i}
\end{align}

Right distributivity follows similarly.

\textbf{(2) Subdistributivity over Intersection}:

For left subdistributivity:
\begin{align}
(a,c) \in \Composition{S}{\left( \bigcap_{i \in I} R_i \right)} 
&\Leftrightarrow \Exists b{(a,b) \in \bigcap_{i \in I} R_i \text{ and } (b,c) \in S} \\
&\Leftrightarrow \Exists b{\left( \Forall i \in I{(a,b) \in R_i} \right) \text{ and } (b,c) \in S} \\
&\Rightarrow \Forall i \in I{\Exists b{(a,b) \in R_i \text{ and } (b,c) \in S}} \\
&\Leftrightarrow \Forall i \in I{(a,c) \in \Composition{S}{R_i}} \\
&\Leftrightarrow (a,c) \in \bigcap_{i \in I} \Composition{S}{R_i}
\end{align}

Right subdistributivity follows similarly. Note that equality doesn't hold in general.

\textbf{(3) Monotonicity}:

If $R_1 \subseteq R_2$:
\begin{align}
(a,c) \in \Composition{S}{R_1} 
&\Leftrightarrow \Exists b{(a,b) \in R_1 \text{ and } (b,c) \in S} \\
&\Rightarrow \Exists b{(a,b) \in R_2 \text{ and } (b,c) \in S} \quad (\text{since } R_1 \subseteq R_2) \\
&\Leftrightarrow (a,c) \in \Composition{S}{R_2}
\end{align}

If $S_1 \subseteq S_2$:
\begin{align}
(a,c) \in \Composition{S_1}{R} 
&\Leftrightarrow \Exists b{(a,b) \in R \text{ and } (b,c) \in S_1} \\
&\Rightarrow \Exists b{(a,b) \in R \text{ and } (b,c) \in S_2} \quad (\text{since } S_1 \subseteq S_2) \\
&\Leftrightarrow (a,c) \in \Composition{S_2}{R}
\end{align}

\textbf{(4) Antimonotonicity with Complement}:

For the first part:
\begin{align}
(a,c) \in \Composition{\ComplementRel {S}}{R} 
&\Leftrightarrow \Exists b{(a,b) \in R \text{ and } (b,c) \in \ComplementRel {S}} \\
&\Leftrightarrow \Exists b{(a,b) \in R \text{ and } (b,c) \notin S} \\
&\Rightarrow \neg \Forall b{(a,b) \in R \Rightarrow (b,c) \in S} \\
&\Leftrightarrow (a,c) \notin \Composition{S}{R} \\
&\Leftrightarrow (a,c) \in \ComplementRel p{\Composition{S}{R}}
\end{align}

For the second part:
\begin{align}
(a,c) \in \Composition{S}{\ComplementRel {R}} 
&\Leftrightarrow \Exists b{(a,b) \in \ComplementRel {R} \text{ and } (b,c) \in S} \\
&\Leftrightarrow \Exists b{(a,b) \notin R \text{ and } (b,c) \in S} \\
&\Rightarrow \neg \Forall b{(b,c) \in S \Rightarrow (a,b) \in R} \\
&\Leftrightarrow (a,c) \notin \Composition{S}{R} \\
&\Leftrightarrow (a,c) \in \ComplementRel p{\Composition{S}{R}}
\end{align}
\end{proof}


\begin{proposition}[Algebraic Properties of Composition for Binary Relations]
Let $r_i = (A, B, R_i)$, $s_i = (B, C, S_i)$, and $t = (C, D, T)$ be binary relations (with matching domains and codomains for composition). Let $\{r_i\}_{i \in I}$ be a family of binary relations from $A$ to $B$, and $\{s_i\}_{i \in I}$ be a family of binary relations from $B$ to $C$. Then:
\begin{enumerate}
    \item \textbf{Distributivity over Union}:
    \begin{itemize}
        \item Left distributivity: $\Composition{s}{\left( \bigcup_{i \in I} r_i \right)} = \bigcup_{i \in I} \Composition{s}{r_i}$
        \item Right distributivity: $\Composition{\left( \bigcup_{i \in I} s_i \right)}{r} = \bigcup_{i \in I} \Composition{s_i}{r}$
    \end{itemize}
    
    \item \textbf{Subdistributivity over Intersection}:
    \begin{itemize}
        \item Left subdistributivity: $\Composition{s}{\left( \bigcap_{i \in I} r_i \right)} \subseteq \bigcap_{i \in I} \Composition{s}{r_i}$
        \item Right subdistributivity: $\Composition{\left( \bigcap_{i \in I} s_i \right)}{r} \subseteq \bigcap_{i \in I} \Composition{s_i}{r}$
    \end{itemize}
    
    \item \textbf{Monotonicity}:
    \begin{itemize}
        \item If $r_1 \subseteq r_2$ then $\Composition{s}{r_1} \subseteq \Composition{s}{r_2}$ (Left monotonicity)
        \item If $s_1 \subseteq s_2$ then $\Composition{s_1}{r} \subseteq \Composition{s_2}{r}$ (Right monotonicity)
    \end{itemize}
    
    \item \textbf{Antimonotonicity with Complement}:
    \begin{itemize}
        \item $\Composition{\ComplementRel{s}}{r} \subseteq \ComplementRel p{\Composition{s}{r}}$
        \item $\Composition{s}{\ComplementRel{r}} \subseteq \ComplementRel p{\Composition{s}{r}}$
    \end{itemize}
    where the complement is taken relative to the same domain and codomain.
\end{enumerate}
\end{proposition}

\begin{proof}
All properties follow from the corresponding properties for the underlying graphs, since composition, union, intersection, and complement of binary relations are defined in terms of their graphs.
\end{proof}

\begin{remark}
The antimonotonicity properties show that composition is "complement-reversing" in each argument. Note that these are inclusions rather than equalities - the reverse inclusions do not hold in general.
\end{remark}

\subsection{Restriction and Corestriction, Direct and Inverse Images}
\begin{definition}[Restriction]
Let $R$ be a graph of a binary relation and $A$ be a set. The \textbf{restriction} of $R$ to $A$, denoted $\Restriction{R}[A]$, is defined as:
\[\Restriction{R}[A] \defeq \left\{ (x, y) \in R \mid x \in A \right\} = R \cap (A \times \Range{R})\]

If $r = (A, B, R)$ is a binary relation, then the restriction of $R$ to $A' \subseteq A$, denoted $\Restriction{r}[A']$, is just $(A', B, \Restriction{R}[A'])$.
\end{definition}

\begin{definition}[Corestriction]
Let $R$ be a graph of a binary relation and $B$ be a set. The \textbf{corestriction} of $R$ to $B$, denoted $\Restriction{R}[][B]$, is defined as:
\[\Restriction{R}[][B] \defeq \left\{ (x, y) \in R \mid y \in B \right\} = R \cap (\Domain{R} \times B)\]

If $r = (A, B, R)$ is a binary relation, then the corestriction of $R$ to $B' \subseteq B$, denoted $\Restriction{r}[][B'] \defeq $, is just $(A, B', \Restriction{R}[][B'])$.
\end{definition}

\begin{remark}
    The notation works to both restrict and corestrict at the same time.
    \[ \Restriction{R}[A][B] \defeq \left\{ (x, y) \in R \mid x \in A \text{ and } y \in B \right\} = R \cap (A \times B) \]

    If $r = (A, B, R)$ is a binary relation and $A' \subseteq A$ and $B' \subseteq B$, denoted $\Restriction{r}[A'][B']$, is just $(A', B', \Restriction{R}[A'][B'])$.
\end{remark}

\begin{definition}[Direct Image]
Let $R$ be a graph of a binary relation and $A$ be a set. The \textbf{direct image} of $A$ under $R$, denoted $\DirectImage{R}{A}$, is defined as:
\[ \DirectImage{R}{A} \defeq \left\{ y \in \Range{R} \mid \Exists{x \in A}{(x, y) \in R} \right\} = \Range{\Restriction{R}[A][]} \]

If $r= (A, B, R)$ is a binary relation, then the direct image of $A$ under $r$, denoted $\DirectImage{r}{A}$, is just $\DirectImage{r}{A} \defeq \DirectImage{R}{A}$.
\end{definition}

\begin{definition}[Inverse Image]
Let $R$ be a graph of a binary relation and $B$ be a set. The \textbf{inverse image} of $B$ under $R$, denoted $\InverseImage{R}{B}$, is defined as:
\[ \InverseImage{R}{B} \defeq \left\{ x \in \Domain{R} \mid \Exists{y \in B}{(x, y) \in R} \right\} = \Domain{\Restriction{R}[][B]} \]

If $r = (A, B, R)$ is a binary relation, then the inverse image of $B$ under $r$, denoted $\InverseImage{r}{B}$, is just $\InverseImage{r}{B} \defeq \InverseImage{R}{B}$.
\end{definition}

\begin{remark}[Composition via Restriction]
For $r = (A,B,R)$, $s = (C,D,S)$ with $B \subseteq C$, we can define:
\[ \Composition{s}{r} \defeq (A,D,\Composition{(\Restriction{S}[\Range{R}][D])}{(\Restriction{R}[A][\Domain{S}])}) \]
This equals $(A,D,\Composition{S}{R})$ since:
\[ \Composition{S}{R} = \Composition{(\Restriction{S}[\Range{R}][D])}{(\Restriction{R}[A][\Domain{S}])} \]
\end{remark}


\begin{proposition}[Unification via Composition]
Let $R$ be a graph of a binary relation and let $A, B$ be sets. Then:
\begin{enumerate}
    \item $\Restriction{R}[A] = \Composition{R}{\text{Id}_A}$
    \item $\Restriction{R}[][B] = \Composition{\text{Id}_B}{R}$
    \item $\DirectImage{R}{A} = \Range{\Composition{R}{\text{Id}_A}}$
    \item $\InverseImage{R}{B} = \Domain{\Composition{\text{Id}_B}{R}}$
\end{enumerate}
\end{proposition}

\begin{proof}
\textbf{(1) Restriction}: 
\begin{align}
\Composition{R}{\text{Id}_A} &= \left\{ (x, z) \mid \Exists{y}{(x, y) \in \text{Id}_A \text{ and } (y, z) \in R} \right\} \\
&= \left\{ (x, z) \mid \Exists{y \in A}{x = y \text{ and } (y, z) \in R} \right\} \\
&= \left\{ (x, z) \mid x \in A \text{ and } (x, z) \in R \right\} \\
&= \Restriction{R}[A]
\end{align}

\textbf{(2) Corestriction}: 
\begin{align}
\Composition{\text{Id}_B}{R} &= \left\{ (x, z) \mid \Exists{y}{(x, y) \in R \text{ and } (y, z) \in \text{Id}_B} \right\} \\
&= \left\{ (x, z) \mid \Exists{y \in B}{(x, y) \in R \text{ and } y = z} \right\} \\
&= \left\{ (x, y) \mid (x, y) \in R \text{ and } y \in B \right\} \\
&= \Restriction{R}[][B]
\end{align}

\textbf{(3) Direct Image}: From property (1):
\begin{align}
\DirectImage{R}{A} &= \left\{ y \mid \Exists{x \in A}{(x, y) \in R} \right\} \\
&= \Range{\Restriction{R}[A]} \\
&= \Range{\Composition{R}{\text{Id}_A}}
\end{align}

\textbf{(4) Inverse Image}: From property (2):
\begin{align}
\InverseImage{R}{B} &= \left\{ x \mid \Exists{y \in B}{(x, y) \in R} \right\} \\
&= \Domain{\Restriction{R}[][B]} \\
&= \Domain{\Composition{\text{Id}_B}{R}}
\end{align}
\end{proof}

\begin{corollary}[Duality via Converse]
The operations of restriction and corestriction are dual via converse, as are direct and inverse images:
\begin{enumerate}
    \item $\Restriction{\ConverseRel{R}}[B] = \ConverseRel{(\Restriction{R}[][B])}$
    \item $\Restriction{\ConverseRel{R}}[][A] = \ConverseRel{(\Restriction{R}[A])}$
    \item $\Restriction{R}[][B] = \ConverseRel{(\Restriction{\ConverseRel{R}}[B])}$
    \item $\Restriction{R}[A] = \ConverseRel{(\Restriction{\ConverseRel{R}}[][A])}$
    \item $\DirectImage{\ConverseRel{R}}{A} = \InverseImage{R}{A}$
    \item $\InverseImage{\ConverseRel{R}}{B} = \DirectImage{R}{B}$
    \item $\DirectImage{R}{A} = \InverseImage{\ConverseRel{R}}{A}$
    \item $\InverseImage{R}{B} = \DirectImage{\ConverseRel{R}}{B}$
\end{enumerate}
\end{corollary}

\begin{proof}
These follow directly from the definitions and properties of converse operations:

\textbf{(1)}: By definition of restriction and converse:
\begin{align}
\Restriction{\ConverseRel{R}}[B] &= \left\{ (x, y) \in \ConverseRel{R} \mid x \in B \right\} \\
&= \left\{ (x, y) \mid (y, x) \in R \text{ and } x \in B \right\} \\
&= \ConverseRel{\left\{ (y, x) \mid (y, x) \in R \text{ and } x \in B \right\}} \\
&= \ConverseRel{(\Restriction{R}[][B])}
\end{align}

\textbf{(2)}: By definition of corestriction and converse:
\begin{align}
\Restriction{\ConverseRel{R}}[][A] &= \left\{ (x, y) \in \ConverseRel{R} \mid y \in A \right\} \\
&= \left\{ (x, y) \mid (y, x) \in R \text{ and } y \in A \right\} \\
&= \ConverseRel{\left\{ (y, x) \mid (y, x) \in R \text{ and } y \in A \right\}} \\
&= \ConverseRel{(\Restriction{R}[A])}
\end{align}

\textbf{(3)} and \textbf{(4)}: These follow from (1) and (2) by applying converse to both sides and using the involution property $\ConverseRel{(\ConverseRel{R})} = R$.

\textbf{(5)}: By definition:
\begin{align}
\DirectImage{\ConverseRel{R}}{A} &= \left\{ y \mid \Exists{x \in A}{(x, y) \in \ConverseRel{R}} \right\} \\
&= \left\{ y \mid \Exists{x \in A}{(y, x) \in R} \right\} \\
&= \InverseImage{R}{A}
\end{align}

\textbf{(6)}: By definition:
\begin{align}
\InverseImage{\ConverseRel{R}}{B} &= \left\{ x \mid \Exists{y \in B}{(x, y) \in \ConverseRel{R}} \right\} \\
&= \left\{ x \mid \Exists{y \in B}{(y, x) \in R} \right\} \\
&= \DirectImage{R}{B}
\end{align}

\textbf{(7)} and \textbf{(8)}: These follow from (5) and (6) by applying converse to both sides and using the involution property $\ConverseRel{(\ConverseRel{R})} = R$.
\end{proof}

\begin{remark}
This proposition shows that restriction, corestriction, direct images, and inverse images are all special cases of composition with identity relations. This unifies these seemingly different operations under the single framework of relation composition, demonstrating the fundamental role of composition in binary relation theory. 

The corollary further shows the elegant duality between these operations via the converse operation, illustrating the deep symmetries inherent in the theory of binary relations. Specifically, corestriction is the restriction of the converse and dually, while direct image is the inverse image of the converse and dually.
\end{remark}

\begin{corollary}[Properties of Direct and Inverse Images]
Let $R$ and $S$ be graphs of binary relations, and let $A, B$ be sets. Then:
\begin{enumerate}
    \item \textbf{Monotonicity in the relation}:
        \begin{itemize}
            \item If $R \subseteq S$ then $\DirectImage{R}{A} \subseteq \DirectImage{S}{A}$
            \item If $R \subseteq S$ then $\InverseImage{R}{B} \subseteq \InverseImage{S}{B}$
        \end{itemize}
    \item \textbf{Monotonicity in the set}:
        \begin{itemize}
            \item If $A_1 \subseteq A_2$ then $\DirectImage{R}{A_1} \subseteq \DirectImage{R}{A_2}$
            \item If $B_1 \subseteq B_2$ then $\InverseImage{R}{B_1} \subseteq \InverseImage{R}{B_2}$
        \end{itemize}
    \item \textbf{Distributivity over union of sets}:
        \begin{itemize}
            \item $\DirectImage{R}{\left( \bigcup_{i \in I} A_i \right)} = \bigcup_{i \in I} \DirectImage{R}{A_i}$
            \item $\InverseImage{R}{\left( \bigcup_{i \in I} B_i \right)} = \bigcup_{i \in I} \InverseImage{R}{B_i}$
        \end{itemize}
    \item \textbf{Subdistributivity over intersection of sets}:
        \begin{itemize}
            \item $\DirectImage{R}{\left( \bigcap_{i \in I} A_i \right)} \subseteq \bigcap_{i \in I} \DirectImage{R}{A_i}$
            \item $\InverseImage{R}{\left( \bigcap_{i \in I} B_i \right)} \subseteq \bigcap_{i \in I} \InverseImage{R}{B_i}$
        \end{itemize}
    \item \textbf{Distributivity over union of relations}:
        \begin{itemize}
            \item $\DirectImage{\left( \bigcup_{i \in I} R_i \right)}{A} = \bigcup_{i \in I} \DirectImage{R_i}{A}$
            \item $\InverseImage{\left( \bigcup_{i \in I} R_i \right)}{B} = \bigcup_{i \in I} \InverseImage{R_i}{B}$
        \end{itemize}
    \item \textbf{Subdistributivity over intersection of relations}:
        \begin{itemize}
            \item $\DirectImage{\left( \bigcap_{i \in I} R_i \right)}{A} \subseteq \bigcap_{i \in I} \DirectImage{R_i}{A}$
            \item $\InverseImage{\left( \bigcap_{i \in I} R_i \right)}{B} \subseteq \bigcap_{i \in I} \InverseImage{R_i}{B}$
        \end{itemize}
    \item \textbf{Antimonotonicity with complement}:
        \begin{itemize}
            \item $\DirectImage{\ComplementRel{R}}{A} \subseteq \ComplementRel{\DirectImage{R}{A}}$
            \item $\InverseImage{\ComplementRel{R}}{B} \subseteq \ComplementRel{\InverseImage{R}{B}}$
        \end{itemize}
\end{enumerate}
\end{corollary}

\begin{proof}
All properties follow from the representation of images as compositions with identity relations and the algebraic properties of composition.

\textbf{(1) Monotonicity in the relation}:
From the representation $\DirectImage{R}{A} = \Range{\Composition{R}{\text{Id}_A}}$ and the right monotonicity of composition, if $R \subseteq S$ then $\Composition{R}{\text{Id}_A} \subseteq \Composition{S}{\text{Id}_A}$, so their ranges satisfy $\DirectImage{R}{A} \subseteq \DirectImage{S}{A}$. Similarly for inverse images using $\InverseImage{R}{B} = \Domain{\Composition{\text{Id}_B}{R}}$ and left monotonicity.

\textbf{(2) Monotonicity in the set}:
If $A_1 \subseteq A_2$, then $\text{Id}_{A_1} \subseteq \text{Id}_{A_2}$. By left monotonicity of composition, $\Composition{R}{\text{Id}_{A_1}} \subseteq \Composition{R}{\text{Id}_{A_2}}$, so their ranges satisfy $\DirectImage{R}{A_1} \subseteq \DirectImage{R}{A_2}$. Similarly for inverse images using right monotonicity.

\textbf{(3) Distributivity over union of sets}:
\begin{align*}
\DirectImage{R}{\left( \bigcup_{i \in I} A_i \right)} 
&= \Range{\Composition{R}{\text{Id}_{\bigcup_{i} A_i}}} \\
&= \Range{\Composition{R}{\left( \bigcup_{i} \text{Id}_{A_i} \right)}} \\
&= \Range{\bigcup_{i} \Composition{R}{\text{Id}_{A_i}}} \\
&= \bigcup_{i} \Range{\Composition{R}{\text{Id}_{A_i}}} \\
&= \bigcup_{i} \DirectImage{R}{A_i}
\end{align*}
Similarly for inverse images using left distributivity of composition.

\textbf{(4) Subdistributivity over intersection of sets}:
\begin{align*}
\DirectImage{R}{\left( \bigcap_{i \in I} A_i \right)} 
&= \Range{\Composition{R}{\text{Id}_{\bigcap_{i} A_i}}} \\
&= \Range{\Composition{R}{\left( \bigcap_{i} \text{Id}_{A_i} \right)}} \\
&\subseteq \Range{\bigcap_{i} \Composition{R}{\text{Id}_{A_i}}} \\
&\subseteq \bigcap_{i} \Range{\Composition{R}{\text{Id}_{A_i}}} \\
&= \bigcap_{i} \DirectImage{R}{A_i}
\end{align*}
Similarly for inverse images using left subdistributivity of composition.

\textbf{(5) Distributivity over union of relations}:
\begin{align*}
\DirectImage{\left( \bigcup_{i \in I} R_i \right)}{A} 
&= \Range{\Composition{\left( \bigcup_{i} R_i \right)}{\text{Id}_A}} \\
&= \Range{\bigcup_{i} \Composition{R_i}{\text{Id}_A}} \\
&= \bigcup_{i} \Range{\Composition{R_i}{\text{Id}_A}} \\
&= \bigcup_{i} \DirectImage{R_i}{A}
\end{align*}
Similarly for inverse images using right distributivity.

\textbf{(6) Subdistributivity over intersection of relations}:
\begin{align*}
\DirectImage{\left( \bigcap_{i \in I} R_i \right)}{A} 
&= \Range{\Composition{\left( \bigcap_{i} R_i \right)}{\text{Id}_A}} \\
&\subseteq \Range{\bigcap_{i} \Composition{R_i}{\text{Id}_A}} \\
&\subseteq \bigcap_{i} \Range{\Composition{R_i}{\text{Id}_A}} \\
&= \bigcap_{i} \DirectImage{R_i}{A}
\end{align*}
Similarly for inverse images using right subdistributivity.

\textbf{(7) Antimonotonicity with complement}:
From the antimonotonicity of composition:
\begin{align*}
\DirectImage{\ComplementRel{R}}{A} 
&= \Range{\Composition{\ComplementRel{R}}{\text{Id}_A}} \\
&\subseteq \Range{\ComplementRel p{\Composition{R}{\text{Id}_A}}} \\
&\subseteq \ComplementRel{\Range{\Composition{R}{\text{Id}_A}}} \\
&= \ComplementRel{\DirectImage{R}{A}}
\end{align*}
Similarly for inverse images using the other antimonotonicity property of composition. Note that the last inclusion holds because for any set $X$, $\Range{\ComplementRel{X}} \subseteq \ComplementRel{\Range{X}}$ when the complement is taken relative to a universal set.
\end{proof}

\begin{proposition}[Properties of Images for Binary Relations]
Let $r = (A, B, R)$ and $s = (A, B, S)$ be binary relations with the same domain and codomain. Then:
\begin{enumerate}
    \item \textbf{Monotonicity in the relation}:
        \begin{itemize}
            \item If $R \subseteq S$ then $\DirectImage{r}{A} \subseteq \DirectImage{s}{A}$
            \item If $R \subseteq S$ then $\InverseImage{r}{B} \subseteq \InverseImage{s}{B}$
        \end{itemize}
    \item \textbf{Monotonicity in the set}:
        \begin{itemize}
            \item If $A_1 \subseteq A_2$ then $\DirectImage{r}{A_1} \subseteq \DirectImage{r}{A_2}$
            \item If $B_1 \subseteq B_2$ then $\InverseImage{r}{B_1} \subseteq \InverseImage{r}{B_2}$
        \end{itemize}
    \item \textbf{Distributivity over union of sets}:
        \begin{itemize}
            \item $\DirectImage{r}{\left( \bigcup_{i \in I} A_i \right)} = \bigcup_{i \in I} \DirectImage{r}{A_i}$
            \item $\InverseImage{r}{\left( \bigcup_{i \in I} B_i \right)} = \bigcup_{i \in I} \InverseImage{r}{B_i}$
        \end{itemize}
    \item \textbf{Subdistributivity over intersection of sets}:
        \begin{itemize}
            \item $\DirectImage{r}{\left( \bigcap_{i \in I} A_i \right)} \subseteq \bigcap_{i \in I} \DirectImage{r}{A_i}$
            \item $\InverseImage{r}{\left( \bigcap_{i \in I} B_i \right)} \subseteq \bigcap_{i \in I} \InverseImage{r}{B_i}$
        \end{itemize}
    \item \textbf{Distributivity over union of relations}:
        \begin{itemize}
            \item $\DirectImage{\left( \bigcup_{i \in I} r_i \right)}{A} = \bigcup_{i \in I} \DirectImage{r_i}{A}$
            \item $\InverseImage{\left( \bigcup_{i \in I} r_i \right)}{B} = \bigcup_{i \in I} \InverseImage{r_i}{B}$
        \end{itemize}
    \item \textbf{Subdistributivity over intersection of relations}:
        \begin{itemize}
            \item $\DirectImage{\left( \bigcap_{i \in I} r_i \right)}{A} \subseteq \bigcap_{i \in I} \DirectImage{r_i}{A}$
            \item $\InverseImage{\left( \bigcap_{i \in I} r_i \right)}{B} \subseteq \bigcap_{i \in I} \InverseImage{r_i}{B}$
        \end{itemize}
    \item \textbf{Antimonotonicity with complement}:
        \begin{itemize}
            \item $\DirectImage{\ComplementRel{r}}{A} \subseteq B \setminus \DirectImage{r}{A}$
            \item $\InverseImage{\ComplementRel{r}}{B} \subseteq A \setminus \InverseImage{r}{B}$
        \end{itemize}
\end{enumerate}
\end{proposition}

\begin{proof}
The first six properties follow directly from the corresponding properties for graphs of binary relations. For the antimonotonicity properties:

Let $r = (A, B, R)$. Then $\ComplementRel{r} = (A, B, (A \times B) \setminus R)$.

For direct image:
\begin{align*}
\DirectImage{\ComplementRel{r}}{A} 
&= \left\{ y \in B \mid \Exists{x \in A}{(x, y) \in (A \times B) \setminus R} \right\} \\
&= \left\{ y \in B \mid \Exists{x \in A}{(x, y) \notin R} \right\} \\
&\subseteq \left\{ y \in B \mid y \notin \DirectImage{r}{A} \right\} \\
&= B \setminus \DirectImage{r}{A}
\end{align*}

Similarly for inverse image:
\begin{align*}
\InverseImage{\ComplementRel{r}}{B} 
&= \left\{ x \in A \mid \Exists{y \in B}{(x, y) \in (A \times B) \setminus R} \right\} \\
&= \left\{ x \in A \mid \Exists{y \in B}{(x, y) \notin R} \right\} \\
&\subseteq \left\{ x \in A \mid x \notin \InverseImage{r}{B} \right\} \\
&= A \setminus \InverseImage{r}{B}
\end{align*}
\end{proof}