
\section{Induced Functions on Power Sets}

Every binary relation $r = (A, B, R)$ induces two functions between the power sets of its domain and codomain.

\begin{proposition}[Induced Functions from Relations]
Let $r = (A, B, R)$ be a binary relation. Then $r$ induces:
\begin{enumerate}
    \item A function $r_* : \mathcal{P}(A) \to \mathcal{P}(B)$ defined by $r_*(N) = \DirectImage{r}{N}$ for all $N \subseteq A$.
    \item A function $r^* : \mathcal{P}(B) \to \mathcal{P}(A)$ defined by $r^*(M) = \InverseImage{r}{M}$ for all $M \subseteq B$.
\end{enumerate}
\end{proposition}

\begin{proof}
Both are well-defined functions since for any subset of the domain/codomain, the direct/inverse image is uniquely determined and is a subset of the codomain/domain respectively.
\end{proof}

\begin{remark}
Not all functions from $\mathcal{P}(A)$ to $\mathcal{P}(B)$ arise from binary relations. We now characterize precisely which functions can be induced by relations.
\end{remark}

\begin{example}[Functions Not Induced by Relations]
Let $A = \{1, 2\}$ and $B = \{a, b\}$. Consider the function $f : \mathcal{P}(A) \to \mathcal{P}(B)$ defined by:
\begin{align*}
f(\emptyset) &= \{a\} \\
f(\{1\}) &= \{b\} \\
f(\{2\}) &= \{a\} \\
f(\{1, 2\}) &= \{a, b\}
\end{align*}

This function cannot be induced by any relation $r = (A, B, R)$ because $f(\emptyset) = \{a\} \neq \emptyset$, but for any relation, $\DirectImage{r}{\emptyset} = \emptyset$.
\end{example}

\begin{proposition}[Characterization of Functions Induced by Relations]
A function $f : \mathcal{P}(A) \to \mathcal{P}(B)$ is induced by some binary relation $r = (A, B, R)$ (i.e., $f = r_*$) if and only if $f$ distributes over arbitrary unions:
\[ f(\bigcup_{i \in I} N_i) = \bigcup_{i \in I} f(N_i) \]
for any family $\{N_i\}_{i \in I}$ of subsets of $A$.
\end{proposition}

\begin{proof}
($\Rightarrow$) Assume $f = r_*$ for some relation $r = (A, B, R)$.

By the properties of direct images established earlier, $\DirectImage{r}{\bigcup_{i \in I} N_i} = \bigcup_{i \in I} \DirectImage{r}{N_i}$.

($\Leftarrow$) Assume $f$ distributes over arbitrary unions. Define $R = \{(a, b) \in A \times B \mid b \in f(\{a\})\}$. We claim that $f = r_*$ where $r = (A, B, R)$.

For any $N \subseteq A$, we can write $N = \bigcup_{a \in N} \{a\}$. Then:
\begin{align*}
\DirectImage{r}{N} &= \{b \in B \mid \Exists :{a \in N}{(a, b) \in R}\} \\
&= \{b \in B \mid \Exists :{a \in N}{b \in f(\{a\})}\} \\
&= \bigcup_{a \in N} f(\{a\}) \\
&= f\left(\bigcup_{a \in N} \{a\}\right) \quad \text{(by distributivity over unions)} \\
&= f(N)
\end{align*}

Thus $f = r_*$.
\end{proof}

\begin{remark}
Note that the property $f(\emptyset) = \emptyset$ is automatically satisfied since taking $I = \emptyset$ in the arbitrary union property gives $f\left(\bigcup_{i \in \emptyset} N_i\right) = f(\emptyset) = \bigcup_{i \in \emptyset} f(N_i) = \emptyset$.
\end{remark}

