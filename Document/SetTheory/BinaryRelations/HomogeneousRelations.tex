
\chapter{Homogeneous Relations}

\section{Definition}

A binary relation $r = (A, A, R)$ where the domain and codomain are the same set is called a \textbf{homogeneous relation} or \textbf{endorelation} on $A$.

Homogeneous relations have special properties that do not apply to general heterogeneous relations. In this section, we explore these properties.

\section{Reflexivity, Symmetry, and Transitivity}

\begin{definition}[Reflexive Relation]
A homogeneous relation $r = (A, A, R)$ is \textbf{reflexive} if every element is related to itself:
\[ \Forall :{a \in A}{(a, a) \in R} \]
Equivalently, $\Identity{A} \subseteq R$.
\end{definition}

\begin{definition}[Irreflexive Relation]
A homogeneous relation $r = (A, A, R)$ is \textbf{irreflexive} if no element is related to itself:
\[ \Forall :{a \in A}{(a, a) \notin R} \]
Equivalently, $R \cap \Identity{A} = \emptyset$.
\end{definition}

\begin{definition}[Symmetric Relation]
A homogeneous relation $r = (A, A, R)$ is \textbf{symmetric} if whenever $a$ is related to $b$, then $b$ is related to $a$:
\[ \Forall :{a, b \in A}{(a, b) \in R \implies (b, a) \in R} \]
Equivalently, $R \subseteq \ConverseRel{R}$ in which case $R = \ConverseRel{R}$.
\end{definition}

\begin{definition}[Asymmetric Relation]
A homogeneous relation $r = (A, A, R)$ is \textbf{asymmetric} if whenever $a$ is related to $b$, then $b$ is not related to $a$:
\[ \Forall :{a, b \in A}{(a, b) \in R \implies (b, a) \notin R} \]
Equivalently, $R \cap \ConverseRel{R} = \emptyset$.
\end{definition}

\begin{definition}[Antisymmetric Relation]
A homogeneous relation $r = (A, A, R)$ is \textbf{antisymmetric} if whenever $a$ is related to $b$ and $b$ is related to $a$, then $a = b$:
\[ \Forall :{a, b \in A}{(a, b) \in R \tand (b, a) \in R \implies a = b} \]
Equivalently, $R \cap \ConverseRel{R} \subseteq \Identity{A}$.
\end{definition}

\begin{definition}[Transitive Relation]
A homogeneous relation $r = (A, A, R)$ is \textbf{transitive} if whenever $a$ is related to $b$ and $b$ is related to $c$, then $a$ is related to $c$:
\[ \Forall :{a, b, c \in A}{(a, b) \in R \tand (b, c) \in R \implies (a, c) \in R} \]
Equivalently, $\Composition{R}{R} \subseteq R$.
\end{definition}

\begin{definition}[Connected Relation]
A homogeneous relation $r = (A, A, R)$ is \textbf{connected} (or \textbf{complete}) if for any two distinct elements, at least one is related to the other:
\[ \Forall :{a, b \in A}{a \neq b \implies (a, b) \in R \tor (b, a) \in R} \]
Equivalently, $R \cup \ConverseRel{R} \cup \Identity{A} = A \times A$.
\end{definition}

\begin{definition}[Strongly Connected Relation]
A homogeneous relation $r = (A, A, R)$ is \textbf{strongly connected} (or \textbf{total}) if for any two elements (not necessarily distinct), at least one is related to the other:
\[ \Forall :{a, b \in A}{(a, b) \in R \tor (b, a) \in R} \]
Equivalently, $R \cup \ConverseRel{R} = A \times A$.
\end{definition}

\begin{definition}[Trichotomous Relation]
A homogeneous relation $r = (A, A, R)$ is \textbf{trichotomous} if for any two elements, exactly one of the following holds: $a$ is related to $b$, $b$ is related to $a$, or $a = b$:
\[ \Forall :{a, b \in A}{(a, b) \in R \oplus (b, a) \in R \oplus a = b} \]
where $\oplus$ denotes exclusive or. Equivalently, $R$, $\ConverseRel{R}$, and $\Identity{A}$ are pairwise disjoint and $R \cup \ConverseRel{R} \cup \Identity{A} = A \times A$.
\end{definition}

\begin{definition}[Minimal Element Property (Non-Strict)]
A homogeneous relation $r = (A, A, R)$ has the \textbf{minimal element property} if every nonempty subset of $A$ has a minimal element with respect to $R$:
\[ \Forall :{S \subseteq A}{S \neq \emptyset \implies \Exists :{m \in S}{\Forall :{s \in S}{(s, m) \in R \implies s = m}}} \]
\end{definition}

\begin{definition}[Minimal Element Property (Strict)]
A homogeneous relation $r = (A, A, R)$ has the \textbf{strict minimal element property} if every nonempty subset of $A$ has a strict minimal element with respect to $R$:
\[ \Forall :{S \subseteq A}{S \neq \emptyset \implies \Exists :{m \in S}{\Forall :{s \in S}{(s, m) \notin R}}} \]
\end{definition}

\begin{definition}[Minimum Element Property (Non-Strict)]
A homogeneous relation $r = (A, A, R)$ has the \textbf{minimum element property} if every nonempty subset of $A$ has a minimum element with respect to $R$:
\[ \Forall :{S \subseteq A}{S \neq \emptyset \implies \Exists :{m \in S}{\Forall :{s \in S}{(m, s) \in R}}} \]
\end{definition}

\begin{definition}[Minimum Element Property (Strict)]
A homogeneous relation $r = (A, A, R)$ has the \textbf{strict minimum element property} if every nonempty subset of $A$ has a strict minimum element with respect to $R$:
\[ \Forall :{S \subseteq A}{S \neq \emptyset \implies \Exists :{m \in S}{\Forall :{s \in S}{s \neq m \implies (m, s) \in R}}} \]
\end{definition}

\section{Special Types of Homogeneous Relations}

\begin{definition}[Preorder Relation]
A homogeneous relation $r = (A, A, R)$ is a \textbf{preorder relation} if it is reflexive and transitive.
\end{definition}

\begin{definition}[Equivalence Relation]
A homogeneous relation $r = (A, A, R)$ is an \textbf{equivalence relation} if it is reflexive, symmetric, and transitive.
\end{definition}

\begin{definition}[Partial Order]
A homogeneous relation $r = (A, A, R)$ is a \textbf{partial order} if it is reflexive, antisymmetric, and transitive.
\end{definition}

\begin{definition}[Strict Partial Order]
A homogeneous relation $r = (A, A, R)$ is a \textbf{strict partial order} if it is irreflexive, asymmetric, and transitive.
\end{definition}

\begin{definition}[Total Order]
A homogeneous relation $r = (A, A, R)$ is a \textbf{total order} (or \textbf{linear order}) if it is a partial order that is also connected. Equivalently, it is reflexive, antisymmetric, transitive, and connected.
\end{definition}

\begin{definition}[Strict Total Order]
A homogeneous relation $r = (A, A, R)$ is a \textbf{strict total order} if it is a strict partial order that is also trichotomous. Equivalently, it is irreflexive, asymmetric, transitive, and trichotomous.
\end{definition}

\begin{remark}
These definitions form a hierarchy of increasingly structured relations:
\begin{itemize}
    \item Equivalence relations partition sets into equivalence classes
    \item Partial orders provide a notion of comparison without requiring all elements to be comparable
    \item Total orders extend partial orders by requiring all elements to be comparable
    \item Well-orders are total orders with the additional property that every subset has a least element
\end{itemize}
\end{remark}



