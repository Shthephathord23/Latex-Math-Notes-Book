% Document/LinearAlgebra/Modules/Modules.tex

\chapter{Modules}

\section{Rings}

\begin{definition}[Ring]
A \textbf{ring} is a set $R$ together with two binary operations, addition ($+$) and multiplication ($\cdot$), usually denoted as a tuple $(R, +, \cdot)$ such that:

\textbf{Addition Structure:}
\begin{enumerate}
    \item $(R, +)$ is an abelian group:
    \begin{itemize}
        \item \textbf{Associativity}: $(a + b) + c = a + (b + c)$ for all $a, b, c \in R$
        \item \textbf{Commutativity}: $a + b = b + a$ for all $a, b \in R$
        \item \textbf{Identity}: $\Exists{0 \in R} \Forall{a \in R} a + 0 = 0 + a = a$
        \item \textbf{Inverses}: $\Forall{a \in R} \Exists{(-a) \in R} a + (-a) = (-a) + a = 0$
    \end{itemize}
\end{enumerate}

\textbf{Multiplication Structure:}
\begin{enumerate}
    \setcounter{enumi}{1}
    \item \textbf{Associativity}: $(ab)c = a(bc)$ for all $a, b, c \in R$
\end{enumerate}

\textbf{Distributive Laws:}
\begin{enumerate}
    \setcounter{enumi}{2}
    \item \textbf{Left distributivity}: $a(b + c) = ab + ac$ for all $a, b, c \in R$
    \item \textbf{Right distributivity}: $(a + b)c = ac + bc$ for all $a, b, c \in R$
\end{enumerate}
\end{definition}

\begin{definition}[Ring with Unity]
A ring $R$ is called a \textbf{ring with unity} (or \textbf{unital ring}) if there exists an element $1 \in R$ such that $1a = a1 = a$ for all $a \in R$. The element $1$ is called the \textbf{multiplicative identity} or \textbf{unity}.
\end{definition}

\begin{example}
Common examples of rings include:
\begin{enumerate}
    \item $\bb{Z}$, $\bb{Q}$, $\bb{R}$, $\bb{C}$ with ordinary addition and multiplication
    \item $\bb{Z}_n$ (integers modulo $n$) for any positive integer $n$
    \item Matrix rings $M_n(R)$ for any ring $R$
    \item Polynomial rings $R[x]$ for any ring $R$
\end{enumerate}
\end{example}

\section{Modules}

\begin{definition}[Left Module]
Let $R$ be a ring (with unity). A \textbf{left $R$-module} (or \textbf{left module over $R$}), denoted as $_R M$, is a set $M$ together with an operation $+ : M \times M \to M$ a scalar multiplication operation $\cdot : R \times M \to M$ such that for all $r, s \in R$ and all $m, n \in M$:

\textbf{Addition Structure:}
\begin{enumerate}
    \item $(M, +)$ is an abelian group:
    \begin{itemize}
        \item \textbf{Associativity}: $(a + b) + c = a + (b + c)$ for all $a, b, c \in M$
        \item \textbf{Commutativity}: $a + b = b + a$ for all $a, b \in M$
        \item \textbf{Identity}: $\Exists :{0 \in M} \Forall :{a \in M} a + 0 = 0 + a = a$
        \item \textbf{Inverses}: $\Forall :{a \in M} \Exists :{(-a) \in M} a + (-a) = (-a) + a = 0$
    \end{itemize}
\end{enumerate}

\textbf{Scalar Multiplication Structure:}
\begin{enumerate}
    \item \textbf{Associativity}: $r(sm) = (rs)m$
    \item \textbf{Unity}: $1 \cdot m = m$ (if $R$ is unital and $1$ is the unity of $R$)
    \item \textbf{Left distributivity}: $r(m + n) = rm + rn$
    \item \textbf{Right distributivity}: $(r + s)m = rm + sm$
\end{enumerate}
\end{definition}

\begin{definition}[Right Module]
Let $R$ be a ring with unity. A \textbf{right $R$-module} (or \textbf{right module over $R$}), denoted as $M_R$, is a set $M$ together with an operation $+ : M \times M \to M$ a scalar multiplication operation $\cdot : R \times M \to M$ such that for all $r, s \in R$ and all $m, n \in M$:

\textbf{Addition Structure:}
\begin{enumerate}
    \item $(M, +)$ is an abelian group:
    \begin{itemize}
        \item \textbf{Associativity}: $(a + b) + c = a + (b + c)$ for all $a, b, c \in M$
        \item \textbf{Commutativity}: $a + b = b + a$ for all $a, b \in M$
        \item \textbf{Identity}: $\Exists :{0 \in M} \Forall :{a \in M} a + 0 = 0 + a = a$
        \item \textbf{Inverses}: $\Forall :{a \in M} \Exists :{(-a) \in M} a + (-a) = (-a) + a = 0$
    \end{itemize}
\end{enumerate}

\begin{enumerate}
    \item \textbf{Associativity}: $(mr)s = m(rs)$
    \item \textbf{Unity}: $m \cdot 1 = m$ (if $R$ is unital and $1$ is the unity of $R$)
    \item \textbf{Left distributivity}: $(m + n)r = mr + nr$
\end{enumerate}
\end{definition}

\begin{remark}
When $R$ is a commutative ring, there is no distinction between left and right modules, and we simply speak of \textbf{$R$-modules}.
\end{remark}

\begin{example}
Important examples of modules include:
\begin{enumerate}
    \item Every vector space over a field $F$ is an $F$-module
    \item Every abelian group is a $\bb{Z}$-module
    \item For any ring $R$, the ring $R$ itself is both a left and right $R$-module
    \item If $R$ is a ring and $I$ is an ideal of $R$, then $R/I$ is an $R$-module
\end{enumerate}
\end{example}
